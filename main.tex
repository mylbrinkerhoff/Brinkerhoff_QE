% !TEX TS-program = lualatex
% !TEX encoding = UTF-8 Unicode		

\documentclass[12pt, letterpaper]{article}

%%BIBLIOGRAPHY- This uses biber/biblatex to generate bibliographies according to the 
%%Unified Style Sheet for Linguistics
\usepackage[main=american, german]{babel}% Recommended
\usepackage{csquotes}% Recommended
\usepackage[backend=biber,
		style=unified,
		maxcitenames=3,
		maxbibnames=99,
		natbib,
		url=false]{biblatex}
\addbibresource{Library.bib}
\setcounter{biburlnumpenalty}{100}  % allow URL breaks at numbers
%\setcounter{biburlucpenalty}{100}   % allow URL breaks at uppercase letters
%\setcounter{biburllcpenalty}{100}   % allow URL breaks at lowercase letters

%%TYPOLOGY
\usepackage[svgnames]{xcolor} % Specify colors by their 'svgnames', for a full list of all colors available see here: http://www.latextemplates.com/svgnames-colors
%\usepackage[compact]{titlesec}
%\titleformat{\section}[runin]{\normalfont\bfseries}{\thesection.}{.5em}{}[.]
%\titleformat{\subsection}[runin]{\normalfont\scshape}{\thesubsection}{.5em}{}[.]
\usepackage[hmargin=1in,vmargin=1in]{geometry}  %Margins
\usepackage{graphicx}	%Inserting graphics, pictures, images
\graphicspath{{Images/}}
\usepackage{stackengine} %Package to allow text above or below other text, Also helpful for HG weights 
\usepackage{fontspec} %Selection of fonts must be ran in XeLaTeX or LuaLaTeX
\usepackage{amssymb} %Math symbols
\usepackage{amsmath} % Mathematical enhancements for LaTeX
\usepackage{setspace} %Linespacing
\usepackage{multicol} %Multicolumn text
\usepackage{enumitem} %Allows for continuous numbering of lists over examples, etc.
\usepackage{multirow} %Useful for combining cells in tables 
\usepackage{booktabs}
\usepackage{hanging}
\usepackage{fancyhdr} %Allows for the 
\pagestyle{fancy}
\fancyhead[L]{\textit{QE draft}} 
\fancyhead[R]{\textit{Brinkerhoff}} 
\fancyfoot[L,R]{} 
\fancyfoot[C]{\thepage} 
\renewcommand{\headrulewidth}{0.4pt}
\setlength{\headheight}{14.5pt} % ...at least 14.49998pt
% \usepackage{fourier} % This allows for using certain wingdings like bombs, frowns, etc.
% \usepackage{fourier-orns} %More useful symbols like bombs and jolly-roger, mostly for OT
\usepackage[colorlinks,allcolors={black},urlcolor={blue}]{hyperref} %allows for hyperlinks and pdf bookmarks
% \usepackage{url} %allows for URLs
% \def\UrlBreaks{\do\/\do-} %allows for URLs to be broken up
\usepackage[normalem]{ulem} %strike out text. Handy for syntax
\usepackage{tcolorbox}
\usepackage{datetime2}
\usepackage{caption}
\usepackage{subcaption}
% \usepackage{titling}
% \setlength{\droptitle}{-2cm}

%%FONTS
\setmainfont{Libertinus Serif}
\setsansfont{Libertinus Sans}
\setmonofont[Scale=MatchLowercase]{Libertinus Mono}

%%PACKAGES FOR LINGUISTICS
%\usepackage{OTtablx} %Generating tableaux with using TIPA
% \usepackage[noipa]{OTtablx} % Use this one generating tableaux without using TIPA
%\usepackage[notipa]{ot-tableau} % Another tableau drawing packing use for posters.
% \usepackage{linguex} % Linguistic examples
% \usepackage{langsci-linguex} % Linguistic examples
\usepackage{langsci-gb4e} % Language Science Press' modification of gb4e
% \usepackage{langsci-avm} % Language Science Press' AVM package
\usepackage{tikz} % Drawing Hasse diagrams
% \usepackage{pst-asr} % Drawing autosegmental features
% \usepackage{pstricks} % required for pst-asr, OTtablx, pst-jtree.
% \usepackage{pst-jtree} 	% Syntax tree drawing software
% \usepackage{tikz-qtree}	% Another syntax tree drawing software. Uses bracket notation.
% \usepackage[linguistics]{forest}	% Another syntax tree drawing software. Uses bracket notation.
% \usepackage{ling-macros} % Various linguistic macros. Does not work with linguex.
% \usepackage{covington} % Another linguistic examples package.
\usepackage{leipzig} %	Offers support for Leipzig Glossing Rules

%%LEIPZIG GLOSSING FOR ZAPOTEC
\newleipzig{el}{el}{elder}	% Elder pronouns
\newleipzig{hu}{hu}{human}	% Human pronouns
\newleipzig{an}{an}{animate}	% Animate pronouns
\newleipzig{in}{in}{inanimate}	% Inanimate pronouns
\newleipzig{pot}{pot}{potential}	% Potential Aspect
\newleipzig{cont}{cont}{continuative}	% Continuative Aspect
% \newleipzig{pot}{pot}{potential}	% Potential Aspect
\newleipzig{stat}{stat}{stative}	% Potential Aspect
\newleipzig{and}{and}{andative}	% Andative Aspect
\newleipzig{ven}{ven}{venative}	% Venative Aspect
% \newleipzig{res}{res}{restitutive}	% Restitutive Aspect
\newleipzig{rep}{rep}{repetitive}	% Repetitive Aspect

%%TITLE INFORMATION
\title{Acoustic discriminability of phonation in Santiago Laxopa Zapotec\thanks{I am grateful to Fe Silva-Robles and  Raúl Díaz Robles for sharing their time and language expertise. I am also grateful to Grant McGuire, Jaye Padgett, Rachel Walker, Maziar Toosarvandani, Ben Eischens, Kim Tan, and Zach Horton for their help and discussions during all stages of this project. This project branched off from collaborating with Jack Duff and Maya Wax Cavallaro.

This work was supported in part by the National Science Foundation under Grant No. 2019804, the Humanities Institute at UC Santa Cruz, and the Jacobs Research Funds.}}
\author{Mykel Loren Brinkerhoff}
\date{\today}

%%MACROS
\newcommand{\sub}[1]{\textsubscript{#1}}
\newcommand{\supr}[1]{\textsuperscript{#1}}
\providecommand{\lsptoprule}{\midrule\toprule}
\providecommand{\lspbottomrule}{\bottomrule\midrule}
\newcommand{\fittable}[1]{\resizebox{\textwidth}{!}{#1}}

\makeatletter
\renewcommand{\paragraph}{%
  \@startsection{paragraph}{4}%
  {\z@}{0ex \@plus 1ex \@minus .2ex}{-1em}%
  {\normalfont\normalsize\bfseries}%
}
\makeatother
\parindent=10pt


\begin{document}
	
%%If using linguex, need the following commands to get the correct LSA style spacing
%% these have to be after  \begin{document}
	% \setlength{\Extopsep}{6pt}
	% \setlength{\Exlabelsep}{9pt}		%effect of 0.4in indent from left text edge

%% Line spacing setting. Comment out the line spacing you do not need. Comment out all if you want single spacing.
    % \doublespacing
    \onehalfspacing

\maketitle
% \thispagestyle{fancy}

\tableofcontents

%------------------------------------
\section{Introduction} \label{sec:Introduction}
%------------------------------------
Non-modal phonation is a common phenomenon in many of the world's languages. Phonation describes how speakers alter the larynx to produce different sound qualities. Most frequently, the larynx is manipulated to produce sounds that vary from breathy to creaky.\footnote{Other types of phonation also exist but are not as frequently employed for linguistic expression (see \cite{eslingVoiceQualityLaryngeal2019} for a detailed discussion on the different phonation types that exist and how the larynx produces them).} In languages such as English, these characteristics are described as being pathological or simply the characteristic of a given speaker (e.g., \cite{klattAnalysisSynthesisPerception1990}). Different phonation types in other languages, such as Gujarati, are used phonemically where vowels can be breathy or modal \citep{fischer-jorgensenPhoneticAnalysisBreathy1968}. This phonemic use of phonation is particularly common in the Oto-Manguean language family from southern Mexico (e.g., \cite{suarezMesoamericanIndianLanguages1983,campbellMesoAmericaLinguisticArea1986,silvermanLaryngealComplexityOtomanguean1997,campbellOtomangueanHistoricalLinguistics2017a,campbellOtomangueanHistoricalLinguistics2017}).

One common problem facing linguists studying phonation is determining the acoustic correlates for these different phonation types. Since \citet{fischer-jorgensenPhoneticAnalysisBreathy1968}, it has been widely assumed that certain markers in the acoustic signal define different types of phonation. The most common measures invoked are spectral slope and harmonics-to-noise ratios. Spectral slope measurements typically involve looking at the relative amplitude of different harmonics in the speech signal. Harmonics-to-noise ratios typically involve looking at the energy in speech signals and are excellent indicators of periodicity. 

Over the years, different authors have proposed different measurements for the phonation types in any given language. For example, \citet{fischer-jorgensenPhoneticAnalysisBreathy1968} proposed that H1-H2, the difference in amplitude between the first and second harmonics, was the best measure for accounting for the difference between breathy and modal vowels in Gujarati. Because phonation is acoustically multi-dimensional, there are many measures that researchers can rely on when they research a given phonation system \citep{garellekAcousticDiscriminabilityComplex2020}. This multitude of acoustic measurements challenges researchers in making cross-linguistic comparisons in defining and describing phonation contrasts. Additionally, knowing which acoustic measures are important for producing and perceiving different phonation types presents a challenge. This paper addresses this challenge by testing and determining which acoustic measures are most relevant for analyzing the four phonation contrasts in Santiago Laxopa Zapotec, an understudied Northern Zapotec language from Oaxaca, Mexico. 

I address this question within the framework proposed by \citet{kreimanUnifiedTheoryVoice2014}. This framework, which the authors call the \textsc{Psychoacoustic Model of Phonation}, provides researchers with a set of acoustic measures for a standardized way to investigate, compare, and theorize how phonation is produced and perceived. Further information about this framework is given in Section~\ref{sec:Background}. By using this framework for investigating the phonation contrasts in Santiago Laxopa Zapotec, I contribute to the question of the efficacy of this framework in linguistic research. Following \citet{garellekAcousticDiscriminabilityComplex2020}, I use a linear discriminant analysis \citep{fisherUseMultipleMeasurements1936} to show that the Psychoacoustic Model of Phonation does explain the phonation contrasts in Santiago Laxopa Zapotec. This analysis shows that various harmonic-to-noise ratios and Strength-of-Excitation contribute the greatest amount of information in discriminating the phonation types. 

The rest of this paper is structured as follows. In Section~\ref{sec:Background}, I first present information about phonation and how linguists have attempted to capture these contrasts acoustically. Section~\ref{sec:SLZ} discusses the phonation contrasts in Santiago Laxopa Zapotec. A general discussion about the elicitation methods and how data for this paper has been processed for the linear discriminant analysis follows in Section~\ref{sec:Methods}. I present the results of the linear discriminant analysis in Section~\ref{sec:LDAResults} and discuss these results in Section~\ref{sec:Discussion}. In Section~\ref{sec:Conclusion}, I discuss the general conclusions of this study and lay out avenues for further study.

%------------------------------------
\section{Background} \label{sec:Background}
%------------------------------------

In studying the phonetics of phonation, linguists have discovered multiple methods of describing and accounting for those contrasts. This is accomplished by using different acoustic markers in the signal. The most frequent type of measure is measuring the spectral slope in the harmonic source. This spectral slope measures the difference in amplitude between harmonics. One of the first linguistics studies that used these spectral slope measures was \cite{fischer-jorgensenPhoneticAnalysisBreathy1968}. \citeauthor{fischer-jorgensenPhoneticAnalysisBreathy1968} showed that the difference between the amplitude of the first and second harmonics (H1-H2) could account for the differences between breathy and modal vowels. Indeed, many early studies found this spectral slope measure is particularly useful in languages with complex phonation systems such as Green Hmong \citep{huffmanMeasuresPhonationType1987,andruskiPhonationTypesProduction2000} and Jalapa Mazatec \citep{silvermanPhoneticStructuresJalapa1995,blankenshipTimeCourseBreathiness1997}.

Further research has continued to rely primarily on H1-H2 to distinguish different types of phonation \citep[e.g.,][]{huffmanMeasuresPhonationType1987,klattAnalysisSynthesisPerception1990}. Over time, various researchers began to observe issues relying on this single spectral slope measure. It has been noted that H2 is very susceptible to formants, sub-glottal pressure, and other factors besides the open quotient of the glottis (see \cite{simpsonFirstSecondHarmonics2012, zhangMechanicsHumanVoice2016,zhangCauseeffectRelationshipVocal2016,chaiH1H2Acoustic2022} for a summary of some of the arguments made against H1-H2). This has led many researchers to propose and use other spectral slope measures besides H1-H2. These include looking at H2 minus the amplitude of the fourth harmonic (H4), H1 minus the amplitude of the harmonic closest to different formants, which are termed A1 through A\textit{n} where $n$ corresponds to the formant the harmonic is closest to. From these studies, several patterns have emerged (see \cite{garellekPhoneticsVoice2019} for an overview). Regardless of which spectral slope measure is used, vowels produced with a breathy phonation have a higher spectral slope measurement when compared to vowels produced with modal phonation, and vowels produced with creaky phonation have a lower spectral slope measurement than modal vowels.  

\begin{figure}[!h]
	\centering
	\includegraphics[width=0.9\textwidth]{Images/Harmonics.png}
	\caption{Spectral slice with LPC smoothed line overlaid for the vowel [e]. Each of the solid peaks represents the harmonics in the spectral slice. The leftmost black solid line peak is the first harmonic (H1), and each subsequent peak represents the next highest harmonic (H2 through H\textit{n}). The red dotted line represents an LPC smoothed line which identifies the formants by the peaks in the line.}
	\label{fig:Harmonics}
\end{figure}

In addition to spectral slope measures, inharmonic source noise measures frequently accompany phonation investigations (e.g., \cite{garellekPhoneticsWhiteHmong2021}). These measures are typically called Harmonic-to-noise ratios. \citet{dekromCepstrumBasedTechniqueDetermining1993} describes these measures as deriving the measurement by liftering the pitch component of the cepstrum and comparing the energy of the harmonics with the noise floor. These measures essentially describe whether or not a sound is periodic or aperiodic within a given bandwidth. For example, HNR\textless 500 Hz is an acoustic measure that describes this harmonic-to-noise ratio between 0 Hz and 500 Hz, with other HNR measures describing the signal less than 1500 Hz, 2500 Hz, or 3500 Hz. These HNR measures are not the only acoustic measures for the inharmonic source noise. Another measure is Cepstral Peak Prominence (CPP; \cite{hillenbrandAcousticCorrelatesBreathy1994,hillenbrandAcousticCorrelatesBreathy1996}). CPP is similar to the harmonics-to-noise ratio measure of \citet{dekromCepstrumBasedTechniqueDetermining1993} but differs in how the ‘prominence’ of the cepstral peak is calculated. Prominence is the difference in decibels between the cepstral peak and a regression line over the cepstrum. In interpreting these measures, a more prominent HNR or CPP value indicates stronger harmonics above the spectrum floor (i.e., greater periodicity in the speech signal). CPP was originally used as a diagnostic for breathy phonation \citep{blankenshipTimingNonmodalPhonation2002,espositoVariationContrastivePhonation2010} In fact, \citeauthor{espositoEffectsLinguisticExperience2010} showed that CPP was the best of the eight measures she considered for distinguishing modal from breathy phonation types. Further research has shown that CPP is also a good measurement for any non-modal phonation (e.g., \cite{andruskiPhonationTypesProduction2000,andruskiToneClarityMixed2006,blankenshipTimingNonmodalPhonation2002,waylandAcousticCorrelatesBreathy2003,avelinoAcousticElectroglottographicAnalyses2010}). 

%------------------------------------
\subsection{Psychoacoustic Model of Phonation} \label{sec:PMoP}
%------------------------------------

These various acoustic measures present researchers with several problems. One of the biggest issues is knowing if all or only a subset of these acoustic measures are necessary for describing and defining phonation contrasts. Another issue this multitude of measures presents is trying to make cross-linguistic comparisons of phonation. If every researcher uses different acoustic measures when describing phonation contrasts, it is nearly impossible to make those cross-linguistic comparisons. Another issue that researchers face is knowing which of these measures are important for the perception of these phonation contrasts in a language. 

The \textsc{Psychoacoustic Model of Phonation} (PMoP; \cite{kreimanUnifiedTheoryVoice2014}) is a theoretical and clinical framework that addresses these problems. The PMoP was designed to assist linguists and clinicians in understanding phonation from the perspectives of production and perception. The goals of this framework are: (i) link perception to acoustics by explaining quality in terms of perceptually valid acoustic measures that combine to determine voice quality fully; (ii) link voice production to acoustics and perception by determining which changes in the physiological voice source produced perceptible changes in the acoustic signal; and (iii) iterate until the two sets of acoustic parameters align. 

\citeauthor{kreimanUnifiedTheoryVoice2014} used these goals to test the many acoustic measures researchers in linguistics and clinical settings have proposed. \citeauthor{kreimanUnifiedTheoryVoice2014} analyzed speech production to determine which acoustic measures produced the most robust measurements. They also conducted perception experiments with synthesized speech, where they manipulated the various acoustic measures using an earlier version of the UCLA Voice Synthesizer \citep{kreimanUCLAVoiceSynthesizer2016} to determine which acoustic measures produced the most robust perceptual cues. 

Based on the results of these experiments and studies, \citeauthor{kreimanUnifiedTheoryVoice2014} proposed that only a subset of measures are necessary for capturing phonation differences. Each measure in this subset falls into one of four different categories that measure the harmonic, inharmonic, time-varying, or vocal tract transfer components; see Table~\ref{tab:Kreiman} for an overview. 

\begin{table}[!h]
    \centering
    \caption{Components of the psychoacoustic model of voice quality and associated parameters (adapted from \cite{kreimanUnifiedTheoryVoice2014,garellekPhoneticsVoice2019}).}
    \label{tab:Kreiman}
    \begin{tabular}{ll}
    \lsptoprule
    Model component & Parameters \\
    \hline
    Harmonic source spectral slope      & H1-H2 \\
                                        & H2-H4 \\
	                                & H4-H2k Hz \\
	                                & H2k Hz-H5k Hz \\
    Inharmonic source noise             & Harmonics-to-noise ratio \\
    Time-varying source characteristics & $f_0$ track \\
	                                & Amplitude track \\
    Vocal tract transfer function       & Formant frequencies and bandwidths \\
	                                & Spectral zeros and bandwidths\\
    \lspbottomrule
    \end{tabular}
\end{table}

There have been two different studies that have tested the validity of this framework for linguistic and clinical settings \citep{garellekAcousticDiscriminabilityComplex2020, kreimanValidatingPsychoacousticModel2021}. Of these two studies \citet{garellekAcousticDiscriminabilityComplex2020} presents a linguistic analysis using this theoretical framework to analyze the six phonation contrasts that exist in !Xóõ. \citeauthor{garellekAcousticDiscriminabilityComplex2020} found that this framework can adequately and reliably be used to describe and define the phonation contrasts for linguistic researchers. One of the goals of this paper is to present another linguistic study validating the use of PMoP for linguistic analysis. I show that the PMoP can adequately provide an account for phonation in the Northern Zapotec language of Santiago Laxopa Zapotec. 

%------------------------------------
\section{Phonation contrasts in Santiago Laxopa Zapotec} \label{sec:SLZ}
%------------------------------------

Santiago Laxopa Zapotec (SLZ; \textit{Dilla'xhunh Laxup} [diʒaˀʐun laʂup]) is a Northern Zapotec language spoken by approximately 1000 people in the municipality of Santiago Laxopa, Ixtlán District in the Sierra Norte of Oaxaca, Mexico \citep{adlerAcousticsPhonationTypes2016,adlerDerivationVerbInitiality2018,foleyForbiddenCliticClusters2018,foleyExtendingPersonCaseConstraint2020}.\footnote{This macro variety is also sometimes called Cajonos Zapotec and comprises the dialects of Zoogocho Zapotec, Yatzachi Zapotec, Yalálag Zapotec, Tabaá Zapotec, and Lachirioag Zapotec \citep{smith-starkAlgunasIsoglosasZapotecas2003}.} It is mutually intelligible with San Bartolomé Zoogocho Zapotec \citep{longDiccionarioZapotecoSan2005,sonnenscheinDescriptiveGrammarSan2005}. SLZ has a fairly standard five-vowel inventory; see Table~\ref{tab:SLZvowels}.\footnote{The /o/ vowel is marginal in the lexicon for SLZ and only appears in a few lexical items. In neighboring San Bartolomé Zoogocho the /u/ vowel is very marginal and has led \citet{sonnenscheinDescriptiveGrammarSan2005} to describe the language as having only four vowels. It is interesting to note that everywhere that SLZ has the vowels /u/ or /o/, Zoogocho only has /o/. When plotting the vowel spaces and looking for outliers in the data based on F1 and F2, I noticed that the vowels /o/ and /u/ occupy nearly identical vowel spaces.}

\begin{table}[!h]
\centering
\caption{Vowels inventory in Santiago Laxopa Zapotec.}
\label{tab:SLZvowels}
    \begin{tabular}{lccc}
    \lsptoprule
	&  front& central  & back \\
    \midrule
    high   	&  i  &     &   u \\
    mid    	&  e  &   	& 	o \\
    low   	&     &  a 	&	  \\
    \lspbottomrule
    \end{tabular}
\end{table}
		
Among Zapotecan languages, it is quite common for languages to make use of contrastive phonation \citep[e.g.,][]{avelinobecerraTopicsYalalagZapotec2004,longDiccionarioZapotecoSan2005,avelinoAcousticElectroglottographicAnalyses2010,lopeznicolasEstudiosFonologiaGramatica2016,chavez-peonInteractionMetricalStructure2010}. In \posscitet{ariza-garciaPhonationTypesTones2018} typological description of phonation in Zapotecan languages, most have two to three phonation types which are described as involving creaky phonation or a glottal closure that sounds similar to a glottal stop in the middle or at end of the vowel. \citet{ariza-garciaPhonationTypesTones2018} additionally notes that breathy phonation is quite rare among Zapotecan languages, with only three languages in her typological study having this phonation type. Based on this typological data, she claims that breathy phonation is a recent innovation and is restricted to the Valley Zapotec languages only. However, SLZ, as a Northern Zapotec language, presents evidence to the contrary. SLZ has a four-way phonation contrast like Valley Zapotecs: modal, breathy, checked, and laryngealized. These contrasts are exemplified in the minimal quadruple in (\ref{ex:YA}).
\ea \label{ex:YA} Four-way near minimal phonation contrast
    \ea \textit{yag}  /çag\supr{L}/ `tree; wood; almúd (unit of measurement approximately 4kg)'
    \ex \textit{yah}  /ça̤\supr{L}/ `metal; rifle; bell'
    \ex \textit{yu'}  /çuˀ\supr{L}/  `earth'
    \ex \textit{ya'a}  /çaˀa\supr{L}/  `market'
    \z 
\z 
In representing the checked and laryngealized vowels, I follow the same procedure as other authors (e.g., \citet{avelinoAcousticElectroglottographicAnalyses2010, uchiharaToneRegistrogenesisQuiavini2016}) in representing the `glottal stop' element as a superscript glottal stop in the IPA transcription (i.e., [aˀ]). This is primarily done as a way of standardizing the variable pronunciation of the glottal element in Zapotec, ranging from a full glottal stop (i.e., [aʔ]) to a creaky portion of the vowel (i.e., [aa̰]).  

Theoretically, one could argue that checked and laryngealized vowels are not actually phonation types but involve a glottal stop consonant, either as a syllabic onset or coda. It is indeed logical that this could be the case. However, much work has shown that this cannot be true in Zapotec. \citet{chavez-peonInteractionMetricalStructure2010} summarizes this work and offers six reasons why these ``glottal stops" are a vocalic feature, not a consonant. The first reason has to do with the distribution of the glottal stop. If it is indeed a coda or onset, then we would expect it could also occur word-initially or in consonant clusters. This does not happen in SLZ or other Zapotec languages \citep{jaegerInitialConsonantClusters1982}. Another point that \citeauthor{chavez-peonInteractionMetricalStructure2010} raises is that both checked and laryngealized vowels show the same tonal contrasts that modal vowels are allowed to have; this is discussed further in Section~\ref{sec:Interaction}. A third point that he raises is that when linguists ask native language consultants about how many syllables or beats the word has, they treat laryngealized vowels as if they only had a single beat.\footnote{This is something Maya Wax Cavallaro and I tested with one of our consultants. The only times that they said one of these laryngealized vowels was not a single beat was when they had differing vowel qualities on either side of the glottal closure (e.g., \textit{bi'a} [biˀa] `fly (insect)').} These and other points raised by \citeauthor{chavez-peonInteractionMetricalStructure2010} are presented in (2). 

\ea  Summary: glottal stop as a vocalic feature in laryngealized and checked vowels (from \cite{chavez-peonInteractionMetricalStructure2010}).
    \ea /ʔ/ defective distribution (not in onset, not in clusters)
    \ex Interrupted vowels have the same tonal sequences as single vowels
    \ex Monosyllabic tendency of the language (roots = 1σ)
    \ex *VʔC\sub{fortis}, predicted by bimoraicity of interrupted vowels
    \ex Same vowel quality, i.e., one vowel gesture (diphthongs a minority)
    \ex Perceived as single syllables by native speakers (ʔ ≠ sufficient consonantal barrier, i.e., syllable boundary)
    \z 
\z 

One point that he doesn't mention is that if we assume the glottal stop is a coda consonant, then we would also expect to see the other phonation types being able to co-occur with this coda consonant. However, there is still an argument to be made that is worth further investigation and study. For this study, I follow tradition in assuming these sounds are vocalic features and contribute to the phonation contrasts. 

Breathy vowels in SLZ are characterized by a raspiness throughout the whole vowel or a portion of the vowel; see Figure~\ref{fig:BreathyVowel}. For some speakers, it appears as if the breathiness is aligned with the beginning of the vowel and others have it aligned to the end of the vowel. 

\begin{figure}[!h]
	\centering
	% [INSERT YAH SPECTROGRAM AND WAVEFORM]
	\includegraphics[width=0.9\textwidth]{Images/yah.png}
	\caption{Breathy vowel in the word \textit{yah} `metal; rifle'}
	\label{fig:BreathyVowel}
\end{figure}

On the other hand, checked vowels are characterized by an abrupt glottal closure which cuts the vowel short. This phonation is sometimes realized as a period of creakiness at the end of the vowel; see Figure~\ref{fig:CheckedVowel}.  

\begin{figure}[!h]
	\centering
	% [INSERT YA SPECTROGRAM AND WAVEFORM]
	\includegraphics[width=0.9\textwidth]{Images/RD_yu'.png}
	\caption{Checked vowel in the word \textit{yu'} `earth'}
	\label{fig:CheckedVowel}
\end{figure}

Laryngealized vowels are common in Zapotecan languages and have received many names. Previous descriptions have used terms such as broken, rearticulated, interrupted, and creaky to describe this phonation type \citep{longDiccionarioZapotecoSan2005,avelinobecerraTopicsYalalagZapotec2004,avelinoAcousticElectroglottographicAnalyses2010,sonnenscheinDescriptiveGrammarSan2005,adlerAcousticsPhonationTypes2016}. To avoid confusion; I will use the term laryngealized following \citet{avelinoAcousticElectroglottographicAnalyses2010}. In addition to their many different names, these vowels exhibit a wide range of allophones. 

\citet{avelinoAcousticElectroglottographicAnalyses2010} found in the closely related Yalálag Zapotec that among his consultants, there were at least four different pronunciations as seen in Table~\ref{tab:laryngeal}. 
\begin{table}[!h]
	\centering
	\caption{Layngealized Vowels in Yalálag Zapotec}
	\label{tab:laryngeal}
	 \begin{tabular}{ll}
	\lsptoprule
	/VˀV/	&  [VʔV]  \\
			&  [VV̰V]   \\
			&  [VV̰ːV̆]  \\
			&  [VV̰V̰]	\\
	\lspbottomrule
	\end{tabular}
\end{table}

In SLZ, this vowel is also highly variable. For most speakers, laryngealized vowels were either creaky throughout their entire production or had a period of creakiness in the middle of the vowel. However, many speakers also had different productions, even when saying the same word. Some of these other productions included producing modal voice throughout, except for a short period of two or three glottal pulses, which showed a drop in amplitude of five to ten decibels. This drop in amplitude is not too surprising as \citet{gerfenProductionPerceptionLaryngealized2005} showed that this drop in amplitude was sufficient to cue these laryngealized vowels in Coatzospan Mixtec, a member of the Amuzgo-Mixtecan branch of the Oto-Manguean language family. Another frequent production was a complete glottal closure in the middle of the vowel producing a true re-articulation of the vowel. In addition to these productions, combinations of these unique productions were also encountered. Based on my observations, these differences cannot be attributed to sociolinguistic factors (e.g., age, sex, gender, socio-economic status). 

To showcase some of these production differences, I show the production of two SLZ speakers who live in Santa Cruz, CA, who participated in piloting this study before I went to Santiago Laxopa for data collection. One of the SLZ speakers in Santa Cruz would re-articulate with a full glottal stop in the middle of the vowel or produce creaky voice. This alternation seemed to be in free variation. Still, there was a greater tendency to creak in low-toned words, such as \textit{xa'ag} [ʂa̰ːg] `topil'\footnote{A \textit{topil} is a type of government office in traditional Oaxacan communities somewhat akin to a sheriff.}, and re-articulate elsewhere; see Figure~\ref{fig:FSRLaryngeal}.

\begin{figure}[!h]
	\centering
	\begin{subfigure}{.5\textwidth}
		\centering
		\includegraphics[width=\linewidth]{Images/za'a.png}
		\caption{\textit{za'a} `corncob'}
		\label{fig:FSRza'a}
	\end{subfigure}%
	\begin{subfigure}{.5\textwidth}
		\centering
		\includegraphics[width=\linewidth]{Images/xa'ag.png}
		\caption{\textit{xa'ag} `topil'}
		\label{fig:FSRxa'ag}
	\end{subfigure}	
	\caption{Comparison of FSR's laryngealized vowels in \textit{za'a} `corncob' and \textit{xa'ag} `topil'}
	\label{fig:FSRLaryngeal}
\end{figure}

The other SLZ speaker only produces creaky voice for these vowels regardless of the tone of the word. During one of the elicitation sessions, my fellow researchers and I conducted a perceptual check that these were, in fact, the same vowels. Both consultants reliably identified the words. They produced laryngealized vowels according to their own idiosyncrasies.
\begin{figure}[!h]
	\centering
	\begin{subfigure}{.5\textwidth}
		\centering
		\includegraphics[width=\linewidth]{Images/RD_za'a.png}
		\caption{\textit{za'a} `corncob'}
		\label{fig:RDza'a}
	\end{subfigure}%
	\begin{subfigure}{.5\textwidth}
		\centering
		\includegraphics[width=\linewidth]{Images/RD_xa'ag.png}
		\caption{\textit{xa'ag} `topil'}
		\label{fig:RDxa'ag}
	\end{subfigure}
	\caption{Comparison of RD's laryngealized vowels in \textit{za'a} `corncob' and \textit{xa'ag} `topil'}
	\label{fig:RDLaryngeal}
\end{figure}

%------------------------------------
\subsection{Interaction of Tone and Phonation} \label{sec:Interaction}
%------------------------------------

Most previous work on the interaction of tone and phonation has been focused on the languages of East and Southeast Asia (e.g., \cite{masicaDefiningLinguisticArea1976,thurgoodVietnameseTonogenesisRevising2002,yipTone2002,enfieldArealLinguisticsMainland2005,michaudComplexTonesEast2012,brunelleTonePhonationSoutheast2016}). What has been found in these descriptions is that certain tones and phonations are codependent. For example, \citet{smalleyProblemsConsonantsTone1976} and \citet{ratliffMeaningfulToneStudy1992} both describe White Hmong's \textit{-g} tone as being a mid-low tone with breathy phonation, and Mandarin's tone 3 is often associated with creaky phonation \citep{hockettPeipingPhonology1947}. \citet{brunelleTonePerceptionNorthern2009} found that creaky phonation plays an important role in producing certain tones. Additionally, work on S'gaw Karen has found that two tones are only differentiated by some form of non-modal phonation (Boehm p.c.). 

However, there have been some observations–especially in Mesoamerica–that tone and phonation are independent of each other \citep[e.g.,][]{silvermanLaryngealComplexityOtomanguean1997,garellekAcousticConsequencesPhonation2011}. This means that tone can independently occur with any phonation type. This has also been extensively described in multiple Zapotecan languages \citep[e.g.,][]{,avelinobecerraTopicsYalalagZapotec2004,avelinoAcousticElectroglottographicAnalyses2010, chavez-peonInteractionMetricalStructure2010, campbellZenzontepecChatinoAspect2011,villardPhonologyMorphologyZacatepec2015, lopeznicolasEstudiosFonologiaGramatica2016}

\citet{chavez-peonInteractionMetricalStructure2010} describes the tone and phonation interactions in San Lucas Quiaviní Zapotec (SLQZ), a central valley variety of Zapotec. The distribution of tone and phonation is found in Table~\ref{tab:SLQZ}. We see that in SLQZ, both low- and falling-tones have the full range of possible combinations. However, we see gaps in the high-tone for breathy and rising tones that can only occur with modal phonation. 

\begin{table}[!ht]
	\centering
	\caption{SLQZ tone and phonation interactions \citep{chavez-peonInteractionMetricalStructure2010}.}
	\label{tab:SLQZ}
	 \begin{tabular}{lcccc}
	  \lsptoprule
					  &	 \textbf{Modal}  & \textbf{Breathy} & \textbf{Creaky} & \textbf{Interrupted} \\
		  High	& ✔︎ & -- & ✔︎ & ✔︎ \\
		  Low & ✔︎ & ✔︎ & ✔︎ & ✔︎ \\
		  Falling & ✔︎ & ✔︎ & ✔︎ & ✔︎ \\
		  Rising & ✔︎ & -- & -- & -- \\
	  \lspbottomrule
	 \end{tabular}
\end{table}

Based on elicitation data collected from 2020-2022, SLZ has a more expansive distribution of tone and phonation when compared to SLQZ but seems to be very similar to other Northern Zapotec varieties \citep[e.g.,][]{avelinobecerraTopicsYalalagZapotec2004}. The distribution of SLZ tonal and phonation combinations are given in Table~\ref{tab:ToneVoiceQuality}.
\begin{table}[!h]
	\caption{SLZ tone and phonation combinations.}
	\label{tab:ToneVoiceQuality}
	\centering

	\begin{tabular}{lcccc}
	\lsptoprule
		& \textbf{Modal} & \textbf{Breathy} & \textbf{Checked} & \textbf{Laryngealized} \\
	\hline
	High		& ✔︎ & -- & ✔︎ & ✔︎ \\
	Mid			& ✔︎ & ✔︎ & ✔︎ & ✔︎ \\
	Low			& ✔︎ & ✔︎ & ✔︎ & ✔︎ \\
	High-Low	& ✔︎ & ✔︎ & ✔︎ & ✔︎ \\
	Mid-High	& ✔︎	& ✔︎ & -- & ✔︎ \\
	\lspbottomrule
	\end{tabular}
\end{table}

One of the striking things in this is the lack of high tone with breathy phonation. This gap is interesting because of the long-time association of high pitch with breathiness \citep[a good overview–of this association and other phonation types–is found in][]{eslingVoiceQualityLaryngeal2019}. This gap is common across the Zapotecan languages that have breathy voice (Campbell p.c.). Regarding breathy phonation in SLQZ, one of the Valley Zapotec varieties, \citet{uchiharaToneRegistrogenesisQuiavini2016} offers some convincing evidence that the phonation originated in syllables with low tone and then spread to other tones via analogy. Investigating the origin of breathy voice in SLZ would be important in understanding how breathy vowels originated in the Zapotecan family where breathy voice is rare typologically \citep{ariza-garciaPhonationTypesTones2018}. Such a study is beyond the scope of this paper.  

%------------------------------------
\subsection{Adler \& Morimoto 2016 } \label{sec:AM2016}
%------------------------------------

\citet{adlerAcousticsPhonationTypes2016} was a previous study of phonation in SLZ. This early study established that there were four phonation types in SLZ and attempted to provide the beginnings of an acoustic analysis of phonation. \citeauthor{adlerAcousticsPhonationTypes2016} had two male SLZ speakers produce carrier sentences containing one of fifty tokens, each appearing five times. This resulted in 500 words used in their analysis. After obtaining this data, they applied every spectral slope measure available in VOICESAUCE \citep{shueVOICESAUCEProgramVoice2009} to the data. They found that for these two male speakers of SLZ, two spectral slope measures robustly captured the contrasts: H1-A1 and H1-A3, which subtract the amplitude of the harmonics closest to the first formant (A1) and third formant (A3) from the amplitude of the first harmonic.

This study was instrumental in providing what contrasts exist in the language. Unfortunately, the results of this study were never published beyond the abstract and, upon further investigation, had several problems with the tokens that they elicited. Several of the words they thought had one type of phonation actually had a different one. Another issue with this study was that they limited their investigation to just spectral slope measures. This means they did not consider any information about the inharmonic source noise, time-varying source characteristics, or vocal tract transfer functions. By relying solely on spectral slope measures, they limited themselves in understanding how to define and describe the phonation contrasts in the language.  

This paper presents a more robust account of SLZ's phonation contrasts by applying \posscitet{kreimanUnifiedTheoryVoice2014} theoretical framework to SLZ. 

%------------------------------------
\section{Methodology} \label{sec:Methods}
%------------------------------------

18 native language speakers of SLZ participated in this study (seven male). In this study, I present data from 6 speakers (three male). All speakers live in Santiago Laxopa Cruz, Ixtlán, Oaxaca, Mexico. The data was collected using a Zoom H4nPro handheld recorder (44.1kHz, 16-bit). Participants were recorded saying approximately 76 words in isolation, and the carrier sentence \textit{shnia' X chonhe lhas} [ʃniaˀ X tʃone ɾas] `I say X three times'. This phrase was repeated three times. 

%------------------------------------
\subsection{Acoustic measurements} \label{sec:Acoustics}
%------------------------------------

The vowels from the target words in carrier sentences were annotated in Praat \citep{boersmaPraatDoingPhonetics2021}. Following \citet{garellekAcousticDiscriminabilityComplex2020}, the vowel onset was set to the second glottal pulse after the beginning of the vowel, and the vowel offset was set to the last glottal pulse before the drop in amplitude of the following consonant or change in formant amplitude when the following consonant was a sonorant (e.g., nasal). 

VoiceSauce \citep{shueVOICESAUCEProgramVoice2009} was used to compute the different acoustic measurements across each vowel. Each vowel was resampled at 16kHZ. The fundamental frequency (\textit{f0}) was calculated using STRAIGHT \citep{kawaharaInstantaneousfrequencybasedPitchExtraction1998}. Formants and bandwidths were calculated using SNACK \citep{sjolanderSnackSoundToolkit2004}.

A total of 1785 vowels were initially considered for analysis. Outliers were removed based on \textit{f0} and vowel formants distance. 28 vowels whose \textit{f0} had a Mahalanobis distance greater than 6 were removed as outliers.\footnote{The Mahalanobis distance is defined as the distance between a point $x$ and a distribution with mean $\mu$ and covariance matrix $\Sigma$. It is calculated using the following formula: $D(x) = \sqrt{ (x - \mu)^T \Sigma^{-1} (x - \mu) }$. It is often used for outlier detection, and points with a significantly large Mahalanobis distance from the distribution are considered potential outliers.} An additional 85 vowels were removed as outliers because their F1 and F2 had a standard deviation greater than 3. This resulted in a total of 1672 vowels being used for this analysis.

Because phonation in SLZ involves dynamic changes (i.e., glottalization occurring in the middle or end of the vowel for laryngealized and checked vowels), each acoustic parameter had three different associated measurements. These three measures are the average over the entire vowel, the change in the parameter from the beginning to the middle of the vowel, and the change in the parameter from the middle to the vowel’s end. These last two measurements will be called ``delta" measurements. This method is not unusual and was used by \citet{garellekAcousticDiscriminabilityComplex2020} for the phonation contrasts in !Xóõ.  

%------------------------------------
\section{Results} \label{sec:Results}
%------------------------------------

%------------------------------------
\subsection{Linear Discriminant Analysis} \label{sec:LDAResults}
%------------------------------------

A Linear Discriminant Analysis (LDA: \cite{fisherUseMultipleMeasurements1936}) is a type of dimension reduction technique in statistics. What this means in practice is that an LDA takes multiple variables as the input to the model and reduces those multiple variables into a few new variables called linear discriminant functions (LD). These LDs are then used to identify which original variables affect the data most. For this study, an LDA was conducted in R \citep{rcoreteamLanguageEnvironmentStatistical2021} using the MASS package from \citet{venablesModernAppliedStatistics2002}.

In the case of SLZ phonation, each of the different acoustic measurements from \posscitet{kreimanUnifiedTheoryVoice2014} were treated as the variables that the LDA used to determine the new LDs. For the statistical model to not be unduly swayed by large numerical differences in the raw numbers, it is standard that all variables are normalized. In this instance, all acoustic measurements proposed by the PMoP were standardized by speaker using a z-score transformation except for the formants and bandwidths, which were standardized by speaker and vowel quality. Strength-of-Excitation (SoE) was standardized by speaker following \citet{garellekVoicingGlottalConsonants2021} where each SoE value was: first log-normalized, then the maximum and minimum log(SoE) values were calculated for each speaker. Finally, the log(SoE) values were subtracted from the minimum value and divided by the difference between the maximum–minimum values.

The LDA model was trained on 70\% of the total data and then tested on the remaining 30\%. The number of LDs produced by the model is always the number of categories you try to predict minus one. Because SLZ has four phonation types, the LDA produced three LDs which accounted for 100\% of the variation in the data. The first linear discriminant (LD1) accounted for 53.24\% of the variation, while the second (LD2) and third (LD3) linear discriminant functions accounted for 30.73\% and 16.03\%, respectively. The confusion matrix from the LDA is shown in Table~\ref{tab:Confusion}; overall, the accuracy of the model's classification of the test set was 70.45\%, much higher than chance at 25\%.

\begin{table}[!h]
    \centering
    \caption{Confusion matrix from the linear discriminate analysis for the testing set.}
    \label{tab:Confusion}
    \begin{tabular}{lllll}
	\lsptoprule
	\begin{tabular}[c]{@{}l@{}}Actual → \\ Predicted ↓\end{tabular} & modal & breathy & checked & laryngealized \\
	\hline
	modal & 294 & 28 & 25 & 24 \\
	breathy & 16 & 32 & 9 & 15 \\
	checked & 13 & 5 & 22 & 3 \\
	laryngealized & 4 & 6 & 3 & 12\\
    \lspbottomrule
    \end{tabular}
\end{table}

Modal vowels represent 64\% of the training data, breathy vowels represent 15.2\%, checked vowels represent 10.8\%, and laryngealized vowels represent 10\%.  Modal vowels were the most accurately discriminated (89.9\%), whereas laryngealized vowels were the least accurately discriminated (22.2\%). All phonation types are most confusable with modal ones. Additionally, laryngealized vowels were frequently confused with breathy vowels.

Within each LD, the different acoustic measures are assigned a coefficient indicating the measure has strength on the LD. When evaluating which measures had the greatest strength you take the absolute value of the coefficient and look for the largest coefficients. This evaluation produced the results reported in Table~\ref{tab:LDA} (only the three strongest measures are shown for each LD).
\begin{table}[!h]
    \centering
    \caption{Acoustic measures with the highest absolute coefficient values for the three linear discriminant functions.}
    \label{tab:LDA}
    \begin{tabular}{lll}
	\lsptoprule
	LD1  & LD2 & LD3  \\
	\hline
    \begin{tabular}[c]{@{}l@{}}HNR \textless 2500 Hz\\ HNR \textless 3500 Hz\\ SoE\end{tabular} & \begin{tabular}[c]{@{}l@{}}HNR \textless 2500 Hz\\ HNR \textless 500 Hz\\ SoE\end{tabular} & \begin{tabular}[c]{@{}l@{}}HNR \textless 2500 Hz\\ HNR \textless 3500 Hz\\ $\Delta$HNR \textless 2500 Hz (end)\end{tabular} \\
    \lspbottomrule
    \end{tabular}
\end{table}

LD1 discriminates modal vowels from the other phonation types and is most strongly associated with HNR \textless 2500 Hz, HNR \textless 3500 Hz, and SoE; see Figure~\ref{fig:LD1}. 
\begin{figure}[!h]
    \centering
    \includegraphics[width=.75\linewidth]{Images/ld1_density.png}
    \caption{Density plot showing LD1 grouped by SLZ's phonation types.}
    \label{fig:LD1}
\end{figure}

LD2 discriminates breathy and checked vowels from the other phonation types and is most strongly associated with HNR \textless 2500 Hz and SoE. HNR \textless 500 Hz is also strongly associated with LD2; see Figure~\ref*{fig:LD2}. 
\begin{figure}[!h]
    \centering
    \includegraphics[width=.75\linewidth]{Images/ld2_density.png}
    \caption{Density plot for LD2 grouped by SLZ's phonation types.}
    \label{fig:LD2}
\end{figure}

LD3 discriminates laryngealized vowels from the other phonation types and is also highly associated with the same measures as LD1 except for SoE, which is instead replaced by $\Delta$HNR \textless 2500 Hz (end), which represents the difference in HNR \textless 2500 Hz from the middle to the end of the vowel; see Figure~\ref*{fig:LD3}. 
\begin{figure}[!h]
    \centering
    \includegraphics[width=.75\linewidth]{Images/ld3_density.png}
    \caption{Density plot for LD3 grouped by SLZ's phonation types.}
    \label{fig:LD3}
\end{figure}


%------------------------------------
\subsection{Statistical results on the output of the LDA} \label{sec:Stats}
%------------------------------------

The five resulting acoustic measures from the LDA were then analyzed for statistical significance through a mixed-effects linear regression using R's lme4 package \citep{batesFittingLinearMixedEffects2015}. The acoustic measures were treated as the dependent variable in each linear regression model, with phonation as the predictor. The interaction between speaker and phonation was treated as a random variable, as was each of the token words.\footnotetext{DV \thicksim Phonation + (Phonation|Speaker) + (1|Word)} When evaluating the mixed-effects linear regression models, the modal vowel was also treated as the intercept of the model. This means the regression model results for each non-modal phonation type are made to the modal vowel.  

The first linear discriminant function was most strongly affected by HNR\textless 2500 Hz, HNR\textless 3500 Hz, and Strength-of-Excitation. The two HNR measures are associated with measures of periodicity but in different bandwidths. HNR\textless 2500 Hz is concerned with measuring periodicity below 2500 Hz. Because non-modal phonation is associated with aperiodicity, we expect the values for breathy, checked, and laryngealized vowels to be lower than those for modal vowels. As we see in Figure~\ref*{fig:HNR25}, the values for breathy, checked, and laryngealized vowels are all lower than the modal vowels. 
\begin{figure}[!h]
	\centering
	\includegraphics[width=.75\linewidth]{Images/HNR25.png}
	\caption{Line graph showing the change of HNR\textless 2500 Hz across the vowels grouped by SLZ's phonation types.}
	\label{fig:HNR25}
\end{figure}

When we turn to the mixed-effects linear regression model for HNR<2500 Hz, we observe that all three values will change negatively to the modal's values for every step of change in modal vowels. However, only the changes in breathy and checked vowels are significant; see Table~\ref*{tab:HNR25}.
\begin{table}[!h]
    \centering
    \caption{Results of the mixed-effects linear regression analysis for HNR < 2500 Hz.}
    \label{tab:HNR25}
    \begin{tabular}{lrrrrrl}
	\lsptoprule
					&  Estimate  & Std. Error & df & t value & p-value & \\
        \textit{Modal}  &  0.1347	& 0.0558 & 151.5648 &  2.414 & 0.01696 & * \\  
  	Breathy   		&  -0.4914 	& 0.1530 &  9.9072  & -3.213 & 0.00939 & ** \\
	Checked    		&  -0.4669  & 0.1769 &  6.6601  & -2.640 & 0.03499 & * \\
	Laryngealized	&  -0.1654  & 0.1810 & 17.8292  & -0.914 & 0.37293 & \\
    \lspbottomrule
    \end{tabular}
\end{table}

When we turn to the acoustic measure HNR<3500 Hz, which measures periodicity below 3500 Hz, we see that the three non-modal phonation types all show lower values than modal vowels; see Figure~\ref*{fig:HNR35}. Similar to HNR<2500 Hz, these lower HNR values indicate that these phonation types show aperiodicity in their production. 
\begin{figure}[!h]
	\centering
	\includegraphics[width=.75\linewidth]{Images/HNR35.png}
	\caption{Line graph showing the change of HNR\textless 3500 Hz across the vowels grouped by SLZ's phonation types.}
	\label{fig:HNR35}
\end{figure}

In contrast to the mixed-effect linear regression model for HNR<2500 Hz, the mixed-effects linear regression for HNR\textless 3500 Hz shows that the differences we observe between the non-modal phonation types and the modal vowels are not statistically significant; see Table~\ref*{tab:HNR35}. 
\begin{table}[!h]
    \centering
    \caption{Results of the mixed-effects linear regression analysis for HNR < 3500 Hz.}
    \label{tab:HNR35}
    \begin{tabular}{lrrrrrl}
	\lsptoprule
					&  Estimate  & Std. Error & df & t value & p-value & \\
        \textit{Modal}  &   0.10018  &  0.05403 & 150.80549 &  1.854 &  0.0657 & . \\  
  	Breathy   		&  -0.33967  &  0.15350 &  9.77806  &-2.213 &  0.0519 & . \\
	Checked    		&  -0.37542  &  0.17313 &  6.62514  &-2.168 &  0.0690 & . \\
	Laryngealized	&  -0.09672  &  0.16999 & 20.16718  &-0.569 &  0.5757 & \\
    \lspbottomrule
    \end{tabular}
\end{table}

The last acoustic measure strongly associated with the first linear discriminant function was Strength-of-Excitation, a measure of the intensity of voicing regardless of the noise in the signal. This SoE measure means that the closer the value is to zero, the less intense the voicing is and indicates a constriction closure to the glottis \citep{mittalStudyEffectsVocal2014,garellekVoicingGlottalConsonants2021}. When we consider the same type of graph showing the SoE values for each of the different phonation types across the vowel, we see that modal vowels have the highest SoE values across the length of the vowel, which is consistent with it lacking a glottal gesture; see Figure~\ref{fig:SOE}. However, what is interesting to see is that checked vowels begin at roughly the same height as modal vowels for SoE. Still, as the length of the vowel increases, the checked vowels progressively have lower and lower SoE values until becoming the lowest at the end of the vowel. Both breathy and laryngealized also exhibit a lower SoE value than modal phonation, which is consistent with having glottal gestures.

\begin{figure}[!h]
	\centering
	\includegraphics[width=.75\linewidth]{Images/SoE.png}
	\caption{Line graph showing the change of Strength-of-Excitation across the vowels grouped by SLZ's phonation types.}
	\label{fig:SOE}
\end{figure}

When evaluating this measure with a mixed-effects linear regression model, we observe that the changes in SoE across all non-modal phonation types are statistically significant; see Table~\ref{tab:SOE}. 

\begin{table}[!h]
    \centering
    \caption{Results of the mixed-effects linear regression analysis for SoE.}
    \label{tab:SOE}
    \begin{tabular}{lrrrrrl}
	\lsptoprule
					&  Estimate  & Std. Error & df & t value & p-value & \\
        \textit{Modal}  &   0.65006  &  0.04328 &  5.33922 & 15.018 & 1.43e-05 & *** \\  
  	Breathy   		&  -0.11532  &  0.03057 &  7.41179 & -3.772 & 0.006265 & ** \\
	Checked    		&  -0.11516  &  0.02230 &  9.38943 & -5.165 & 0.000517 & *** \\
	Laryngealized	&  -0.08573  &  0.03299 & 11.87923 & -2.599 & 0.023424 & * \\
    \lspbottomrule
    \end{tabular}
\end{table}

The second linear discriminant function was similarly associated with HNR\textless 2500 Hz and SoE like they were for the first linear discriminant function. Instead of HNR\textless 3500 Hz, a different HNR measure strongly affected this second linear discriminant function. HNR\textless 500 Hz measure periodicity below 500 Hz. Figure~\ref{fig:HNR05} shows that, like the other HNR measures, all three non-modal phonation types show a lower HNR value consistent with them being aperiodic. However, when the mixed-effects linear regression model was run on the acoustic measure, the only checked phonation reached significance; see Table~\ref{tab:HNR05}. 

\begin{figure}[!h]
	\centering
	\includegraphics[width=.75\linewidth]{Images/HNR05.png}
	\caption{Line graph showing the change of HNR < 500 Hz across the vowels grouped by SLZ's phonation types.}
	\label{fig:HNR05}
\end{figure}

\begin{table}[!h]
    \centering
    \caption{Results of the mixed-effects linear regression analysis for HNR < 500 Hz.}
    \label{tab:HNR05}
    \begin{tabular}{lrrrrrl}
	\lsptoprule
					&  Estimate  & Std. Error & df & t value & p-value & \\
        \textit{Modal}  &  0.14980  & 0.07553 & 9.64369 &  1.983 & 0.07652 & . \\  
  	Breathy   		&  -0.05906 & 0.20426 & 6.38671 & -0.289 & 0.78162 &  \\
	Checked    		&  -0.70904 & 0.16811 & 6.77873 & -4.218 & 0.00424 & ** \\
	Laryngealized	&  -0.29020 & 0.23947 & 7.58130 & -1.212 & 0.26197 & \\
    \lspbottomrule
    \end{tabular}
\end{table}

The third linear discriminant function only introduced one new acoustic measure. $ \Delta $HNR<2500 Hz (end) measures the difference between the middle and end of the vowel for HNR\textless 2500 Hz. As seen in Figure~\ref{fig:HNR25}, the HNR measure changes most drastically from the middle of the vowel to the end of the vowel for checked phonation. The other phonation types also change but do not change as quickly. The difference can be seen in Figure~\ref{fig:HNR25} when we compare the values in the middle portion of the graph to those at the end for each phonation type. This magnitude of change is reflected in the linear regression for $ \Delta $ HNR<2500 Hz (end), with checked phonation being the only one that showed actual statistical significance; see Table~\ref{tab:delta}. 

\begin{table}[!h]
    \centering
    \caption{Results of the mixed-effects linear regression analysis for $ \Delta $HNR<2500 Hz (end).}
    \label{tab:delta}
    \begin{tabular}{lrrrrrl}
	\lsptoprule
					&  Estimate  & Std. Error & df & t value & p-value & \\
        \textit{Modal}  &   -0.02397 &   0.06574  & 32.69794 & -0.365 & 0.71774 & \\  
  	Breathy   		&    0.12279 &   0.14446  & 11.41768 &  0.850 & 0.41280 & \\
	Checked    		&    0.32531 &   0.10589  & 41.57481 &  3.072 & 0.00374 & ** \\
	Laryngealized	&   -0.12616 &   0.27805  &  8.21579 & -0.454 & 0.66176 & \\
    \lspbottomrule
    \end{tabular}
\end{table}

%------------------------------------
\section{Discussion} \label{sec:Discussion}
%------------------------------------

%------------------------------------
\subsection{Linear discriminant analysis discussion} \label{sec:DiscussionLDA}
%------------------------------------
The results of the linear discriminant analysis showed that the acoustic measures used in the PMoP are sufficient to account for the different phonation types in SLZ. The linear discriminant analysis performed well above chance at 70.45\%. The linear discriminant analysis further showed that each of the four could be discriminated by the three discriminant functions it produced. However, as mentioned above in Section~\ref{sec:LDAResults}, the linear discriminant analysis performed better at classifying some phonation types over others. The model could classify modal vowels best at almost 90\% accuracy; breathy vowels at 45\%; checked at 37\%; and laryngealized at 22\%. The confusion matrix in Table~\ref{tab:Confusion} shows that all nonmodal phonation types were predominantly confused with modal vowels. This is not too unsurprising because of the highly variable nature in which these nonmodal phonation types are produced. 

The linear discriminant analysis additionally showed that the acoustic measures that contributed the most to the three linear discriminant functions were harmonic-to-noise ratios along three different bandwidths and strength-of-excitation. Even though these acoustic measures had the greatest association, it does not mean that these are the only measures necessary to understand the phonation contrasts. Each linear discriminant function comprises a linear combination of all the acoustic measures and their associated coefficient. In more phonological terms, we can imagine these linear discriminant functions similar to the formula phonologists use to derive each candidate's harmonic score within Harmonic Grammar \citep{smolenskyHarmonicMindNeural2006}. In Harmonic Grammar, each constraint contributes weight to the formula, with some constraints contributing a greater weight and others a lower weight; a linear discriminant function assigns a numerical coefficient to each acoustic measure. Just because one acoustic measure has a lower coefficient does not mean an acoustic measure is not important to how speakers produce or perceive the phonation contrasts in Zapotec. A combination of several low-weighted acoustic measures may produce enough ``weight" to overcome the ``weight" of these heavier harmonic-to-noise ratios, similar to gang effects in Harmonic Grammar.\footnote{The exact formulas for each linear discriminant function, which  show the ``weights" for each acoustic measure, can be found in the supplementary materials located online at \url{www.mlbrinkerhoff.me/LDA.html}.}

Instead of using the absolute value of the coefficients in the linear discriminant functions to determine which acoustic measures are to consider, another method is to take the mean predictive value of the three linear discriminant functions and determine the correlation values for each acoustic measure within each linear discriminant function. This is the method used by \citet{garellekAcousticDiscriminabilityComplex2020} for determining which acoustic measures to use in his analysis for !Xóõ. 

Both methods have their merits, but using the coefficients of the acoustic measures instead of their correlations is in line with more general practices used in the statistical literature. Different acoustic measures have the strongest correlation when using this other method with this study's linear discriminant analysis. The three highest acoustic measures according to absolute correlation score for each linear discriminant function are reported in Table~\ref{tab:LDA2}. Evaluating these measures will be left for further analysis. 

\begin{table}[!h]
    \centering
    \caption{Acoustic measures with the highest absolute correlations for the three linear discriminant functions.}
    \label{tab:LDA2}
    \begin{tabular}{lll}
	\lsptoprule
	LD1  & LD2 & LD3  \\
	\hline
    \begin{tabular}[c]{@{}l@{}}SoE \\ $f0$ \\ HNR < 1500 Hz\end{tabular} & \begin{tabular}[c]{@{}l@{}}$f0$\\ $\Delta$SoE (end)\\ B1\end{tabular} & \begin{tabular}[c]{@{}l@{}}$\Delta$SoE (end)\\ $\Delta$F3 (end)\\ $\Delta$HNR \textless 1500 Hz (beg)\end{tabular} \\
    \lspbottomrule
    \end{tabular}
\end{table}

The linear discriminant analysis further shows that the acoustic measures found in the Psychoacoustic Model of Phonation (PMoP; \cite{kreimanUnifiedTheoryVoice2014}) can generate three linear discriminant functions that provide separation of SLZ's four types. This validates this framework for linguistic analysis of phonation. However, as discussed below in Section~\ref{sec:DiscussionLaryngealized}, there are some issues with the acoustic measure H1*-H2*. 
%------------------------------------
\subsection{Harmonic-to-noise ratio discussion} \label{sec:DiscussionHNR}
%------------------------------------

As previously mentioned, the acoustic measures that contributed the most ``weight" to the three linear discriminant functions were predominantly harmonic-to-noise ratios across three different bandwidths. Out of the three harmonic-to-noise ratios the linear discriminant functions called out, all were highly significant for at least one non-modal phonation, except for HNR\textless 3500 Hz. This acoustic measure never reached significance; however, the \textit{p}-value for the breathy and checked non-modal phonations hovered very close to the alpha value of 0.05. HNR\textless 3500 Hz was not as informative because it subsumes the narrower bandwidths of \textless 500 Hz and \textless 2500 Hz with its wider bandwidth of \textless 3500 Hz. It is also important to remember that under the PMoP, a single acoustic measure is not more important than another, but the acoustic measures associated with the harmonic spectral slopes, the inharmonic source noise, the time-varying source characteristics, and the vocal tract transfer functions combined together tell a complete story of how phonation is produced in a language \citep{kreimanUnifiedTheoryVoice2014, garellekPhoneticsVoice2019,kreimanValidatingPsychoacousticModel2021}. 

When relying on the coefficient values, the ``delta" score contributed little to the linear discriminant functions. This result is surprising, considering how the different phonation types behave dynamically, especially when distinguishing between laryngealized and checked vowels. One reason for this could be the same sample size reported in this paper of six speakers. It is very likely that the data from the remaining 12 speakers could improve the model and show the importance of these ``delta" scores. However, when we consider the correlation of the mean predictive value of the linear discriminant function and the acoustic measures instead of the ``weights", as shown in Table~\ref{tab:LDA2} above, these ``delta" scores make up the three highest correlations for the third linear discriminant function and the second highest correlation in the second linear discriminant function. This suggests that depending on evaluating by correlation of the linear discriminant functions and their component variables or evaluating the raw ``weight" of the acoustic measures could produce different outcomes. Accounting for why this is beyond the scope of this paper but will be left for further research. 

%------------------------------------
\subsection{Laryngealized vowel discussion} \label{sec:DiscussionLaryngealized}
%------------------------------------

When we focus on how well the linear discriminant analysis performed at classifying the different acoustic measures, we notice that the model did very poorly for laryngealized vowels, which the linear discriminant analysis performed the worst on at 22\% chance, which is below chance. As mentioned above in Section~\ref{sec:SLZ}, laryngealized vowels show the greatest variability in their production inter- and intra-speaker. For some productions of this vowel, speakers would produce a vowel with model phonation throughout its entirety. The only indication that the vowel was non-model was a small period of two to three glottal pulses that had a lower amplitude than the rest of the vowel, very similar to what \citet{gerfenProductionPerceptionLaryngealized2005} found for laryngealized vowels in Mixtec. At other times, the production was with a tense voice rather than a canonical creaky voice. 

Besides being the most confused with modal vowels, laryngealized vowels were frequently confused with breathy vowels. This result is surprising because laryngealized vowels are not described as breathy within the language and across the Zapotecan languages (see \cite{ariza-garciaPhonationTypesTones2018} for a detailed typology of phonation contrasts in the Zapotecan branch of the Oto-Manguean language family). The linear discriminant analysis only revealed SoE and HNR acoustic measures contributing the most to the linear discriminant functions. As mentioned above, just because an acoustic measure contributes a smaller ``weight" to the functions does not mean it is unimportant to any analysis. These measurements still contribute some amount of weight and should still be considered. 

The PMoP states that a combination of spectral slope and inharmonic noise measures is necessary for describing and identifying the different phonation contrasts in a language. Because HNR measures periodicity, breathy and laryngealized vowels will have lower values concerning these measures because they are both aperiodic. According to \citet{garellekPhoneticsVoice2019}, spectral slope measures and harmonic-to-noise are necessary to classify phonation types. The PMoP suggests that four spectral slope measures must be considered when making claims about the phonation contrasts in a given language. H1*-H2* is by far the most common measure that is reported. As seen in Figure~\ref{fig:H1H2}, which is the H1*-H2* measure averaged across all speakers, laryngealized vowels have a higher spectral slope than the modal vowel in the final fourth of the vowel. This suggests that laryngealized vowels are indeed breathy in a portion of their production. 

\begin{figure}[!h]
	\centering
	\includegraphics[width=.75\linewidth]{Images/h1h2.png}
	\caption{Line graph showing the change of H1*-H2* across the vowels grouped by SLZ's phonation types.}
	\label{fig:H1H2}
\end{figure}

However, we also observe that breathy vowels are nearly identical in H1*-H2*, which begins with a lower measure than the modal vowel until about 75\%  of the way through the vowel when it rises above the modal's values. We should expect to see the breathy vowel having a higher spectral slope measure throughout the duration of the vowel. This could be a problem with the H1*-H2* measure itself, which a growing body of literature argues is a very poor measure of phonation because it is highly sensitive to changes in sub-glottal pressure and to what vowel is being measured. In fact \citet{simpsonFirstSecondHarmonics2012,garellekPhoneticsVoice2019,chaiH1H2Acoustic2022} have all called for no longer using H1*-H2* as an acoustic measure and proposed alternatives. One of the frequent alternatives to H1*-H2* is to use H1* minus one of the harmonics closest to the formants (labeled A1, A2, or A3 where $n$ refers to which formant number). Following \citet{espositoVariationContrastivePhonation2010}, who found H1*-A3* to be a very robust measure in Zapotecan languages, Figure~\ref{fig:H1A3} shows the H1*-A3* measure for SLZ. 
\begin{figure}[!h]
	\centering
	\includegraphics[width=.75\linewidth]{Images/h1a3.png}
	\caption{Line graph showing the change of H1*-A3* across the vowels grouped by SLZ's phonation types.}
	\label{fig:H1A3}
\end{figure}

This measure accurately captures the behavior of the different phonation contrasts in the language. As seen in Figure~\ref{fig:H1A3}, laryngealized vowels show a period of creakiness in the middle of the vowel but additionally show that these same vowels are also breathy in the last fourth of the vowel. This finding supports what was found during segmentation: for many speakers, the portion of the vowel following the drop in amplitude or complete glottal closure sounded breathy. This means that this phonation type is a complex one that begins creaky and breathy, similar to the complex phonation type of breathy-creaky in !Xóõ \citep{garellekAcousticDiscriminabilityComplex2020}.

% \begin{figure}[!h]
% 	\centering
% 	\includegraphics[width=.75\linewidth]{Images/h4h2k.png}
% 	\caption{Line graph showing the change of H4*-H2k Hz* across the vowels grouped by SLZ's phonation types.}
% 	\label{fig:H4H2k}
% \end{figure}

% \begin{figure}[!h]
% 	\centering
% 	\includegraphics[width=.75\linewidth]{Images/h2kh5k.png}
% 	\caption{Line graph showing the change of H2k Hz*-H5k Hz* across the vowels grouped by SLZ's phonation types.}
% 	\label{fig:H2kH5k}
% \end{figure}

% \begin{figure}
%     \centering
%     \begin{subfigure}{.5\textwidth}
% 		\centering
% 		\includegraphics[width=\linewidth]{Images/h1h2.png}
% 		\caption{\textit{xa'ag} `topil'}
% 		\label{fig:H1H2}
% 	\end{subfigure}%
%     \begin{subfigure}{.5\textwidth}
% 		\centering
% 		\includegraphics[width=\linewidth]{Images/h2h4.png}
% 		\caption{\textit{xa'ag} `topil'}
% 		\label{fig:H2H4}
% 	\end{subfigure}%
%     \\
%     \begin{subfigure}{.5\textwidth}
% 		\centering
% 		\includegraphics[width=\linewidth]{Images/h4h2k.png}
% 		\caption{\textit{xa'ag} `topil'}
% 		\label{fig:H4H2k}
% 	\end{subfigure}%
%     \begin{subfigure}{.5\textwidth}
% 		\centering
% 		\includegraphics[width=\linewidth]{Images/h2kh5k.png}
% 		\caption{\textit{xa'ag} `topil'}
% 		\label{fig:H2kH5k}
% 	\end{subfigure}
%     \caption{Comparison of PMoP's four spectral slope measures for SLZ grouped by phonation types. }
% 	\label{fig:SpectralSlopes}
% \end{figure}
%------------------------------------
\subsection{Laryngeal Complexity Hypothesis} \label{sec:LCH}
%------------------------------------
Another question this study provides the groundwork for is in investigating the predictions from the \textsc{Laryngeal Complexity Hypothesis} (LCH), which has its origins in work by \citet{silvermanLaryngealComplexityOtomanguean1997,blankenshipTimeCourseBreathiness1997,blankenshipTimingNonmodalPhonation2002}. The basic premise of the LCH is that in languages that have both tone and phonation, there needs to be an ordering between the two laryngeal gestures for tone and phonation for optimal perceptibility. This premise comes from the understanding that the same mechanism responsible for tone is also responsible for the production of phonation: the vocal folds and glottis. The rate at which vocal folds vibrate is responsible for changing the fundamental frequency, perceived as pitch, which is grammaticalized as tone in lexical tone languages. The faster the vocal folds vibrate, the higher the pitch. And the slower the vocal folds vibrate, the lower the pitch. 

The vocal folds are also the primary articulator for phonation. \citet{ladefogedPreliminariesLinguisticPhonetics1971,gordonPhonationTypesCrosslinguistic2001} treated phonation as a by-product of how open or closed the glottis is during vocal fold vibration. This is schematized by Figure~\ref{fig:Phonation}. The more open the glottis is, the breathier the sound to the point where the sound becomes the sound [h]. The more closed the glottis is, the creakier the sound is to the point where the sound becomes [ʔ]. 

\begin{figure}[!ht]
	\centering
	\includegraphics[width=.6\textwidth]{Phonation.png}
	\caption{One-dimensional model for phonation. Based on \citet{ladefogedPreliminariesLinguisticPhonetics1971,gordonPhonationTypesCrosslinguistic2001}}.
	\label{fig:Phonation}
\end{figure}

The LCH assumes phonation and tone are produced at the vocal folds and glottis. Because these same organs are responsible for these two phenomena, \citet{silvermanLaryngealComplexityOtomanguean1997} argues that there is an articulatory problem in producing non-modal phonation and tone. Not only is this a problem for speakers but is also a problem for listeners who need to perceive the signals for tone and phonation and when non-modal phonation is produced at the same time as tone the signal is very difficult for the listener to recover. It is assumed by the LCH that because of this issue, there needs to be strict ordering in the glottal gestures. This means the tonal gesture must be produced before or after the phonation gesture. The reason for this is that if the gestures overlap, there will be a perturbation of the tone, and the listeners will not be able to differentiate the tone reliably. The LCH assumes that there is a close link between production and perception. This assumption places the responsibility on making sure the acoustic cues are the most perceptually salient on the speaker. The speaker is responsible for producing tone and phonation so that the listener can differentiate the different cues for both tone and phonation. 

In Figure~\ref{fig:GlottalGestures}, which is taken from \citet{dicanioCoarticulationToneGlottal2012}, the cue for tone is represented by the Pitch Target and the Glottal Gesture represents the gestures needed to produce phonation. When the Pitch Target is not co-articulated with the Glottal Gestures, there is the greatest perceptual recoverability for the listener. The LCH argues for this as the tone is most recoverable on modal vowels. This modal portion is then ordered or phased relative to the non-modal portion. This ordering is the speaker's responsibility to accommodate the listeners' perceptibility. If, however, the Pitch Target and the Glottal Gesture were overlapping, as represented by the lower half in Figure~\ref{fig:GlottalGestures}, then the cues for pitch and phonation would be overlapping and making it perceptually more difficult for the listener to recover the signals for tone and phonation.  
\begin{figure}[!ht]
	\centering
	\includegraphics[width=.5\textwidth]{Gestures.png}
	\caption{Representation taken from \citet{dicanioCoarticulationToneGlottal2012}.}
	\label{fig:GlottalGestures}
\end{figure}

Some work has been done investigating this in other languages, most notably \posscitet{dicanioCoarticulationToneGlottal2012} investigation into glottals in Itunyoso Trique, which is also an Oto-Manguean language. \citeauthor{dicanioCoarticulationToneGlottal2012} in his study found that when there was a large amount of coarticulation between the glottal consonants and the vowels, there is a strong correlation between the magnitude of overlap and the amount of perturbation in the $f0$ signal. If the degree of overlap was minor, then the acoustic signal had little to no perturbation.

Another study on Jalapa Mazatec \citep{garellekAcousticConsequencesPhonation2011} also investigated the interaction of tone and phonation. Jalapa Mazatec is a language with both contrastive tone and phonation, and \citet{garellekAcousticConsequencesPhonation2011} validated the claims made by the LCH, in that tone and phonation seemed to be ordered with each other when it comes to at least one of the phonation types. 

SLZ is ideal for testing the LCH because it uses contrastive tone and phonation. Additionally, there is some evidence that sequencing plays a vital role in differentiating checked and laryngealized vowels based on where in the vowel there is a glottal closure. Additionally, evidence from the breathy vowels produced by some speakers speaks to this question because only a portion of the vowel was breathy, with some speakers producing breathy voice on the first or second half of the vowel. 

Another question this study raises is the status of laryngealized vowels and their realizations. We observed from just two speakers that these vowels are highly variable. Assuming that \posscitet{avelinoAcousticElectroglottographicAnalyses2010} observations on this vowel in the closely related Yalálag Zapotec hold in SLZ, there is considerably more variation that exists within and between speakers. Suppose all of these varying realizations of a phonological category are evident. What does this mean for this category's status and the cues necessary for speakers to produce and realize this category? 

%------------------------------------
\section{Conclusions and further directions } \label{sec:Conclusion}
%------------------------------------

In conclusion, this paper presents the first study for SLZ using the PMoP framework. Through a linear discriminant analysis, I show that the acoustic measures proposed by the PMoP are sufficient to distinguish the four phonation types for SLZ. Depending on how one analyzes the linear discriminant functions, either strength-of-excitation and harmonic-to-noise ratios or strength-of-excitation, differences in $f0$, the bandwidth of the first formant, and changes in the frequency of the third formant are the most important for SLZ. Because of the high confusability of laryngealized vowels in the linear discriminant analysis, I show that laryngealized vowels are best analyzed as a complex phonation type that begins creaky and ends breathy. Additionally, this paper contributes to the typological understanding of Zapotecan languages by describing phonation in an understudied family member. I further show that breathy phonation is not restricted to the Valley Zapotec languages but also exists in Northern Zapotec languages. 

This study raises several questions, forming a cohesive study path for future work. This study presents the more immediate question: Are the acoustic measures strongly correlated with the linear discriminant function statistically significant? As briefly discussed above, two different methods for analyzing the linear discriminant functions provide different perspectives on the acoustic measures contributing to the linear discriminant functions—focusing on the LDA coefficients, representing the weights assigned to each predictor variable in the discriminant functions. These coefficients indicate the relative importance of each variable for phonation type separation. Looking at the correlation between the acoustic measures and the mean predictor values of the linear discriminant functions provides an additional perspective. The correlations examine the relationship between each class's acoustic measures and the discriminant functions' central tendency (mean). It helps identify how closely related the acoustic measures are to the mean values of the linear discriminant functions. It can also reveal if certain variables have a stronger or weaker association with the discriminant functions. Addressing this question provides us with a complete understanding of the production of phonation in SLZ. Three closely related questions need to be addressed based on the results of this study and the answers to this immediate question about the statistical significance of the acoustic measure correlates.

The first of these questions concerns the perception of the phonation contrasts in SLZ. Understanding how phonation is produced is only half of the picture. To fully understand these contrasts in the language, a categorical perception experiment where the acoustic measures from the PMoP are manipulated is necessary. This categorical perception experiment would consist of me synthesizing tokens for identification and discrimination by SLZ native speakers. 

Another question raised by the production study in this paper and this perception experiment is: what are the phonological representations for these phonation contrasts? Because these phonation contrasts are productive and necessary for Zapotec, they need to belong to be distinguished in the phonology of the language. I would investigate this question by exploring this question from a phonetic-phonology interface. A gestural account using Gestural Phonology \citep{browmanArticulatoryGesturesPhonological1989,browmanArticulatoryPhonologyOverview1992} may be the most effective framework for addressing this question. 

The third question forming the next fork of study is how phonation in SLZ contributes to our understanding of the Laryngeal Complexity Hypothesis \citep{silvermanLaryngealComplexityOtomanguean1997, blankenshipTimeCourseBreathiness1997, blankenshipTimingNonmodalPhonation2002}. The production data from this paper would be valuable in testing the claims made in this hypothesis, as would the perception data from the first question. From a production side of investigating this hypothesis, I would conduct a generalized additive mixed model (GAMM; \cite{hastieGeneralizedAdditiveModels1986, woodGeneralizedAdditiveModels2017}) on the acoustic measures with the highest coefficients and highest correlation from the linear discriminant analysis. GAMMs are ideally suited for exploring the question of phasing because these statistical models were designed for analyzing dynamic data (i.e., data that involves differences in time) and have been shown to account for phonetic data exceptionally well from both production and perception studies \citep{winterHowAnalyzeLinguistic2016,arnholdModellingInterplayMultiple2017,royIndividualDifferencesPatterns2017,soskuthyGeneralisedAdditiveMixed2017,wielingAnalyzingDynamicPhonetic2018,gonzalezGridlinesApproachDynamic2021,soskuthyEvaluatingGeneralisedAdditive2021,coetzeeProducingPerceivingSocially2022}. 

Together these three questions provide us with an opportunity to use data from an understudied language to provide evidence for the validity of the PMoP for psychoacoustic research in linguistics by providing a better understanding of how phonological contrasts are produced and perceived by speakers and hearers. This also allows us to offer a phonological account of how speakers of laryngeally complex languages represent these phonation contrasts. Finally, this research path would provide a robust framework to test the predictions and explanations for the Laryngeal Complexity Hypothesis. 
%------------------------------------
%BIBLIOGRAPHY
%------------------------------------

%\singlespacing
% \nocite{*}
\printbibliography[heading=bibintoc]

\end{document}