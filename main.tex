% !TEX TS-program = lualatex
% !TEX encoding = UTF-8 Unicode		

\documentclass[12pt, letterpaper]{article}

%%BIBLIOGRAPHY- This uses biber/biblatex to generate bibliographies according to the 
%%Unified Style Sheet for Linguistics
\usepackage[main=american, german]{babel}% Recommended
\usepackage{csquotes}% Recommended
\usepackage[backend=biber,
		style=unified,
		maxcitenames=3,
		maxbibnames=99,
		natbib,
		url=false]{biblatex}
\addbibresource{Library.bib}
\setcounter{biburlnumpenalty}{100}  % allow URL breaks at numbers
%\setcounter{biburlucpenalty}{100}   % allow URL breaks at uppercase letters
%\setcounter{biburllcpenalty}{100}   % allow URL breaks at lowercase letters

%%TYPOLOGY
\usepackage[svgnames]{xcolor} % Specify colors by their 'svgnames', for a full list of all colors available see here: http://www.latextemplates.com/svgnames-colors
%\usepackage[compact]{titlesec}
%\titleformat{\section}[runin]{\normalfont\bfseries}{\thesection.}{.5em}{}[.]
%\titleformat{\subsection}[runin]{\normalfont\scshape}{\thesubsection}{.5em}{}[.]
\usepackage[hmargin=1in,vmargin=1in]{geometry}  %Margins
\usepackage{graphicx}	%Inserting graphics, pictures, images
\graphicspath{{Images/}}
\usepackage{stackengine} %Package to allow text above or below other text, Also helpful for HG weights 
\usepackage{fontspec} %Selection of fonts must be ran in XeLaTeX or LuaLaTeX
\usepackage{amssymb} %Math symbols
\usepackage{amsmath} % Mathematical enhancements for LaTeX
\usepackage{setspace} %Linespacing
\usepackage{multicol} %Multicolumn text
\usepackage{enumitem} %Allows for continuous numbering of lists over examples, etc.
\usepackage{multirow} %Useful for combining cells in tables 
\usepackage{booktabs}
\usepackage{hanging}
\usepackage{fancyhdr} %Allows for the 
\pagestyle{fancy}
\fancyhead[L]{\textit{QE draft}} 
\fancyhead[R]{\textit{Brinkerhoff}} 
\fancyfoot[L,R]{} 
\fancyfoot[C]{\thepage} 
\renewcommand{\headrulewidth}{0.4pt}
\setlength{\headheight}{14.5pt} % ...at least 14.49998pt
% \usepackage{fourier} % This allows for using certain wingdings like bombs, frowns, etc.
% \usepackage{fourier-orns} %More useful symbols like bombs and jolly-roger, mostly for OT
\usepackage[colorlinks,allcolors={black},urlcolor={blue}]{hyperref} %allows for hyperlinks and pdf bookmarks
% \usepackage{url} %allows for URLs
% \def\UrlBreaks{\do\/\do-} %allows for URLs to be broken up
\usepackage[normalem]{ulem} %strike out text. Handy for syntax
\usepackage{tcolorbox}
\usepackage{datetime2}
\usepackage{caption}
\usepackage{subcaption}
% \usepackage{titling}
% \setlength{\droptitle}{-2cm}

%%FONTS
\setmainfont{Libertinus Serif}
\setsansfont{Libertinus Sans}
\setmonofont[Scale=MatchLowercase]{Libertinus Mono}

%%PACKAGES FOR LINGUISTICS
%\usepackage{OTtablx} %Generating tableaux with using TIPA
% \usepackage[noipa]{OTtablx} % Use this one generating tableaux without using TIPA
%\usepackage[notipa]{ot-tableau} % Another tableau drawing packing use for posters.
% \usepackage{linguex} % Linguistic examples
% \usepackage{langsci-linguex} % Linguistic examples
\usepackage{langsci-gb4e} % Language Science Press' modification of gb4e
% \usepackage{langsci-avm} % Language Science Press' AVM package
\usepackage{tikz} % Drawing Hasse diagrams
% \usepackage{pst-asr} % Drawing autosegmental features
% \usepackage{pstricks} % required for pst-asr, OTtablx, pst-jtree.
% \usepackage{pst-jtree} 	% Syntax tree drawing software
% \usepackage{tikz-qtree}	% Another syntax tree drawing software. Uses bracket notation.
% \usepackage[linguistics]{forest}	% Another syntax tree drawing software. Uses bracket notation.
% \usepackage{ling-macros} % Various linguistic macros. Does not work with linguex.
% \usepackage{covington} % Another linguistic examples package.
\usepackage{leipzig} %	Offers support for Leipzig Glossing Rules

%%LEIPZIG GLOSSING FOR ZAPOTEC
\newleipzig{el}{el}{elder}	% Elder pronouns
\newleipzig{hu}{hu}{human}	% Human pronouns
\newleipzig{an}{an}{animate}	% Animate pronouns
\newleipzig{in}{in}{inanimate}	% Inanimate pronouns
\newleipzig{pot}{pot}{potential}	% Potential Aspect
\newleipzig{cont}{cont}{continuative}	% Continuative Aspect
% \newleipzig{pot}{pot}{potential}	% Potential Aspect
\newleipzig{stat}{stat}{stative}	% Potential Aspect
\newleipzig{and}{and}{andative}	% Andative Aspect
\newleipzig{ven}{ven}{venative}	% Venative Aspect
% \newleipzig{res}{res}{restitutive}	% Restitutive Aspect
\newleipzig{rep}{rep}{repetitive}	% Repetitive Aspect

%%TITLE INFORMATION
\title{Acoustic discriminability of phonation in Santiago Laxopa Zapotec\thanks{I am grateful to Fe Silva-Robles and  Raúl Díaz Robles for sharing their time and language expertise. I am also grateful to Grant McGuire, Jaye Padgett, Rachel Walker, Maziar Toosarvandani, Ben Eischens, Kim Tan, and Zach Horton for their help and discussions during all stages of this project. This project branched off from collaborating with Jack Duff and Maya Wax Cavallaro.

This work was supported in part by the National Science Foundation under Grant No. 2019804, the Humanities Institute at UC Santa Cruz, and the Jacobs Research Funds.}}
\author{Mykel Loren Brinkerhoff}
\date{\today}

%%MACROS
\newcommand{\sub}[1]{\textsubscript{#1}}
\newcommand{\supr}[1]{\textsuperscript{#1}}
\providecommand{\lsptoprule}{\midrule\toprule}
\providecommand{\lspbottomrule}{\bottomrule\midrule}
\newcommand{\fittable}[1]{\resizebox{\textwidth}{!}{#1}}

\makeatletter
\renewcommand{\paragraph}{%
  \@startsection{paragraph}{4}%
  {\z@}{0ex \@plus 1ex \@minus .2ex}{-1em}%
  {\normalfont\normalsize\bfseries}%
}
\makeatother
\parindent=10pt


\begin{document}
	
%%If using linguex, need the following commands to get the correct LSA style spacing
%% these have to be after  \begin{document}
	% \setlength{\Extopsep}{6pt}
	% \setlength{\Exlabelsep}{9pt}		%effect of 0.4in indent from left text edge

%% Line spacing setting. Comment out the line spacing you do not need. Comment out all if you want single spacing.
    % \doublespacing
    \onehalfspacing

\maketitle
% \thispagestyle{fancy}

\tableofcontents

%------------------------------------
\section{Introduction} \label{sec:Introduction}
%------------------------------------
Non-modal phonation is a common phenomenon in many of the world's languages. Phonation describes how speakers alter the larynx to produce different sound qualities. Most frequently, the larynx is manipulated to produce sounds that vary from being more breathy or creaky.\footnote{Other types of phonation also exist but are not as frequently employed for linguistic expression (see \cite{eslingVoiceQualityLaryngeal2019} for a detailed discussion on the different phonation types that exist and how the larynx produces them).} In languages such as English, these characteristics are described as being pathological or simply the characteristic of a given speaker (e.g., \cite{klattAnalysisSynthesisPerception1990}). In other languages such as Gujarati, these phonation types are used phonemically where vowels can be breathy or modal \citep{fischer-jorgensenPhoneticAnalysisBreathy1968}. This phonemic use of phonation is particularly common in the Oto-Manguean language family from southern Mexico (e.g., \cite{suarezMesoamericanIndianLanguages1983,campbellMesoAmericaLinguisticArea1986,silvermanLaryngealComplexityOtomanguean1997,campbellOtomangueanHistoricalLinguistics2017a,campbellOtomangueanHistoricalLinguistics2017}).

One common problem facing linguists studying phonation is determining the acoustic correlates for these different phonation types. Since \citet{fischer-jorgensenPhoneticAnalysisBreathy1968}, it has been widely assumed that certain markers in the acoustic signal define different types of phonation. The most common measurements invoked are spectral-tilt measurements and harmonics-to-noise ratios. Spectral-tilt measurements typically involve looking at the relative amplitude of different harmonics in the speech signal. Harmonics-to-noise ratios typically involve looking at how much energy there is in that same speech signal and are typically excellent indicators of periodicity. 

Over the years, different authors have proposed different measurements for the phonation types in any given language. For example, \citet{fischer-jorgensenPhoneticAnalysisBreathy1968} proposed that H1-H2, the difference in amplitude between the first and second harmonics, was the best measurement for accounting for the difference between breathy and modal vowels in Gujarati. Because phonation is acoustically multi-dimensional, there are many measurements that researchers can rely on when they research a given phonation system \citep{garellekAcousticDiscriminabilityComplex2020}. This multitude of acoustic measurements presents a challenge for researchers to make cross-comparisons between and come to a common understanding of what defines a phonation type. Additionally, with many measurements, knowing which measurements are important for the production and perception of different phonation types presents a challenge. One question I address in this paper is the challenge of knowing what measurements are most relevant for analyzing the four phonation contrasts in Santiago Laxopa Zapotec, an understudied Northern Zapotec language from Oaxaca, Mexico. 

I address this question within the framework proposed by \citet{kreimanUnifiedTheoryVoice2014}. This framework, which the authors call the \textsc{Psychoacoustic Model for Phonation}, provides researchers with a set of acoustic measures for a standardized way to investigate, compare, and theorize how phonation is produced and perceived. Further information about this framework is given in Section~\ref{sec:Background}. By using this framework for investigating the phonation contrasts in Santiago Laxopa Zapotec, I contribute to the question of the efficacy of this framework in linguistic research. Following \citet{garellekAcousticDiscriminabilityComplex2020}, I use a linear discriminant analysis \citep{fisherUseMultipleMeasurements1936} to show that the Psychoacoustic Model of Phonation does explain the phonation contrasts in Santiago Laxopa Zapotec. This analysis shows that various harmonic-to-noise ratios and Strength-of-Excitation contribute the greatest amount of information in discriminating the phonation types. 

The rest of this paper is structured as follows. In Section~\ref{sec:Background}, I first present information about phonation and how linguists have attempted to capture these contrasts acoustically. Section~\ref{sec:SLZ} discusses the phonation contrasts in Santiago Laxopa Zapotec. A general discussion about the elicitation methods and how data for this paper has been processed for the linear discriminant analysis follows in Section~\ref{sec:Methods}. I present the results of the linear discriminant analysis in Section~\ref{sec:LDAResults} and discuss these results in Section~\ref{sec:Discussion}.

%------------------------------------
\section{Background} \label{sec:Background}
%------------------------------------

In studying the phonetics of phonation contrasts, linguists have discovered multiple methods of describing and accounting for those contrasts. One of the primary ways this is accomplished is by using different acoustic markers in the signal. The most frequent type of measurement is spectral-tilt. Spectral-tilt measurements are the difference in amplitude between harmonics. One of the first studies using spectral-tilt was \cite{fischer-jorgensenPhoneticAnalysisBreathy1968}. \citeauthor{fischer-jorgensenPhoneticAnalysisBreathy1968} showed that the difference between the amplitude of the first and second harmonics (H1-H2) could account for the differences between breathy and modal vowels. Indeed, many early studies found these spectral-tilt measurements are particularly useful in languages with complex phonation systems such as Green Hmong \citep{huffmanMeasuresPhonationType1987,andruskiPhonationTypesProduction2000} and Jalapa Mazatec \citep{silvermanPhoneticStructuresJalapa1995,blankenshipTimeCourseBreathiness1997}.

Further research has continued to rely primarily on H1-H2 to distinguish different types of phonation \citep[e.g.,][]{huffmanMeasuresPhonationType1987,klattAnalysisSynthesisPerception1990}. Over time, various researchers began experimenting with whether or not the difference between other harmonics besides H1-H2. Over time, other researchers have proposed using other measurements besides H1-H2. These include looking at H2 minus the amplitude of the fourth harmonic (H4) or the amplitude of the harmonic closest to the different formants, which are termed A1 through A\textit{n} where the number corresponds to which formant the harmonic is closest to.  

\begin{figure}[!h]
	\centering
	\includegraphics[width=0.9\textwidth]{Images/Harmonics.png}
	\caption{Spectral slice with LPC smoothed line overlaid for the vowel [e]. Each of the solid peaks represents the harmonics in the spectral slice. The leftmost black solid line peak is the first harmonic (H1), and each subsequent peak represents the next highest harmonic (H2 through H\textit{n}). The red dotted line represents an LPC smoothed line which identifies the formants by the peaks in the line.}
	\label{fig:Harmonics}
\end{figure}

From these studies, several patterns have emerged (see \cite{garellekPhoneticsVoice2019} for an overview). When interpreting the results of these spectral-tilt measures, several general patterns correspond to the different phonation types. For example, vowels produced with a breathy phonation typically have a higher spectral-tilt measurement when compared to vowels produced with modal phonation, and vowels produced with creaky phonation typically have a lower spectral-tilt measurement than modal vowels. 

In addition to spectral-tilt measurements, noise measurements frequently accompany phonation investigations (e.g., \cite{garellekPhoneticsWhiteHmong2021}). 

One of the most commonly used of these noise measurements is Cepstral Peak Prominence (CPP; \cite{hillenbrandAcousticCorrelatesBreathy1994,hillenbrandAcousticCorrelatesBreathy1996}). CPP is similar to the harmonics-to-noise ratio measure of \citet{dekromCepstrumBasedTechniqueDetermining1993} but differs in how the ‘prominence’ of the cepstral peak is calculated. Prominence is taken as the difference in amplitude of the cepstral peak, and a regression line is used to normalize window size and overall energy. In interpreting the measurement, a more prominent cepstral peak indicates stronger harmonics above the spectrum floor (i.e., greater periodicity in the speech signal). CPP was originally used as a diagnostic for breathy or modal voice \citep{blankenshipTimingNonmodalPhonation2002,espositoVariationContrastivePhonation2010} In fact, \citeauthor{espositoEffectsLinguisticExperience2010} showed that CPP was the best of the eight measures she considered for distinguishing modal from breathy phonation types. Further research has shown that CPP is also a good measurement for any non-modal phonation (e.g., \cite{andruskiPhonationTypesProduction2000,andruskiToneClarityMixed2006,blankenshipTimingNonmodalPhonation2002,waylandAcousticCorrelatesBreathy2003,avelinoAcousticElectroglottographicAnalyses2010}). 

%------------------------------------
\subsection{Psychoacoustic Model for Phonation} \label{sec:PMoP}
%------------------------------------


\begin{table}[!h]
    \centering
    \caption{Components of the psychoacoustic model of voice quality and associated parameters (from \cite{kreimanUnifiedTheoryVoice2014}).}
    \label{tab:Kreiman}
    \begin{tabular}{ll}
    \lsptoprule
    Model component & Parameters \\
    \hline
    Harmonic source spectral slope      & H1-H2 \\
                                        & H2-H4 \\
	                                & H4-H2 kHz \\
	                                & H2 kHz-H5 kHz \\
    Inharmonic source noise             & Harmonics-to-noise ratio \\
    Time-varying source characteristics & $f_0$ track \\
	                                & Amplitude track \\
    Vocal tract transfer function       & Formant frequencies and bandwidths \\
	                                & Spectral zeros and bandwidths\\
    \lspbottomrule
    \end{tabular}
\end{table}


%------------------------------------
\section{Phonation contrasts in Santiago Laxopa Zapotec} \label{sec:SLZ}
%------------------------------------

Santiago Laxopa Zapotec (SLZ; \textit{Dilla'xhunh Laxup}) is a Northern Zapotec language spoken by approximately 1000 people in the municipality of Santiago Laxopa, Ixtlán District in the Sierra Norte of Oaxaca, Mexico \citep{adlerAcousticsPhonationTypes2016,adlerDerivationVerbInitiality2018,foleyForbiddenCliticClusters2018,foleyExtendingPersonCaseConstraint2020}. It is mutually intelligible with San Bartolomé Zoogocho Zapotec \citep{longDiccionarioZapotecoSan2005,sonnenscheinDescriptiveGrammarSan2005}. SLZ has a fairly standard five-vowel inventory; see Table~\ref{tab:SLZvowels}.

\begin{table}[!h]
\centering
\caption{Vowels inventory in Santiago Laxopa Zapotec.}
\label{tab:SLZvowels}
    \begin{tabular}{lccc}
    \lsptoprule
	&  front& central  & back \\
    \midrule
    high   	&  i  &     &   u \\
    mid    	&  e  &   	& 	o \\
    low   	&     &  a 	&	  \\
    \lspbottomrule
    \end{tabular}
\end{table}
		
Among Zapotecan languages, it is quite common for languages to make use of contrastive phonation \citep[e.g.,][]{avelinobecerraTopicsYalalagZapotec2004,longDiccionarioZapotecoSan2005,avelinoAcousticElectroglottographicAnalyses2010,lopeznicolasEstudiosFonologiaGramatica2016,chavez-peonInteractionMetricalStructure2010}. 
SLZ, in addition to the five vowel qualities mentioned above, has four contrastive phonation types: modal, breathy, checked, and laryngealized. These contrasts are exemplified in the minimal quadruple in (\ref{ex:YA}).

In representing the checked and laryngealized vowels, I follow the same procedure as other authors (e.g., \citet{avelinoAcousticElectroglottographicAnalyses2010, uchiharaToneRegistrogenesisQuiavini2016}) in representing the `glottal stop' element as a superscript glottal stop in the IPA transcription (i.e., [aˀ]). This is primarily done as a way of standardizing the variable pronunciation of the glottal element in Zapotec, ranging from a full glottal stop (i.e., [aʔ]) to a creaky portion of the vowel (i.e., [aa̰]).  

\ea \label{ex:YA} Four-way near minimal phonation contrast
    \ea \textit{yag}  /çag\supr{L}/ `tree; wood; almúd (unit of measurement approximately 4kg)'
    \ex \textit{yah}  /ça̤\supr{L}/ `metal; rifle; bell'
    \ex \textit{yu'}  /çuˀ\supr{L}/  `earth'
    \ex \textit{ya'a}  /çaˀa\supr{L}/  `market'
    \z 
\z 




Breathy phonation on vowels is characterized by a raspy quality throughout the whole vowel or a portion toward the end of the vowel; see Figure~\ref{fig:BreathyVowel}. 

\begin{figure}[!h]
	\centering
	% [INSERT YAH SPECTROGRAM AND WAVEFORM]
	\includegraphics[width=0.9\textwidth]{Images/yah.png}
	\caption{Breathy vowel in the word \textit{yah} `metal; rifle'}
	\label{fig:BreathyVowel}
\end{figure}

On the other hand, checked vowels are characterized by an abrupt glottal closure which cuts the vowel short. This phonation is sometimes only realized as a very short period of creakiness at the end of the vowel; see Figure~\ref{fig:CheckedVowel}.  

\begin{figure}[!h]
	\centering
	% [INSERT YA SPECTROGRAM AND WAVEFORM]
	\includegraphics[width=0.9\textwidth]{Images/RD_yu'.png}
	\caption{Checked vowel in the word \textit{yu'} `earth'}
	\label{fig:CheckedVowel}
\end{figure}

Laryngealized vowels are common in Zapotecan languages and have received many names. Previous descriptions have used terms such as broken, rearticulated, interrupted, and creaky to describe this phonation type \citep{longDiccionarioZapotecoSan2005,avelinobecerraTopicsYalalagZapotec2004,avelinoAcousticElectroglottographicAnalyses2010,sonnenscheinDescriptiveGrammarSan2005,adlerAcousticsPhonationTypes2016}. To avoid confusion; I will use the term laryngealized following \citet{avelinoAcousticElectroglottographicAnalyses2010}. In addition to many different names, these vowels also exhibit a wide range of allophones. 

\citet{avelinoAcousticElectroglottographicAnalyses2010} found in the closely related Yalálag Zapotec that among his consultants, there were at least four different pronunciations as seen in Table~\ref{tab:laryngeal}. 
\begin{table}[!h]
	\centering
	\caption{Layngealized Vowels in Yalálag Zapotec}
	\label{tab:laryngeal}
	 \begin{tabular}{ll}
	\lsptoprule
	/VˀV/	&  [VʔV]  \\
			&  [VV̰V]   \\
			&  [VV̰ːV̆]  \\
			&  [VV̰V̰]	\\
	\lspbottomrule
	\end{tabular}
\end{table}
In SLZ, each consulted language expert would produce this vowel differently. One consultant would re-articulate, with a full glottal stop in the middle of the vowel, or they would produce creaky voice. This alternation seemed to be in free variation, but there was a greater tendency to creak in low-toned words, such as \textit{xa'ag} [ʂa̰ːg] `topil'\footnote{A \textit{topil} is a type of government office in traditional Oaxacan communities somewhat akin to a sheriff.}, see Figure~\ref{fig:FSRLaryngeal} for a comparison between this consultant's pronunciation of the laryngealized vowels.

\begin{figure}[!h]
	\centering
	\begin{subfigure}{.5\textwidth}
		\centering
		\includegraphics[width=\linewidth]{Images/za'a.png}
		\caption{\textit{za'a} `corncob'}
		\label{fig:za'a}
	\end{subfigure}%
	\begin{subfigure}{.5\textwidth}
		\centering
		\includegraphics[width=\linewidth]{Images/xa'ag.png}
		\caption{\textit{xa'ag} `topil'}
		\label{fig:xa'ag}
	\end{subfigure}	
	\caption{Comparison of FSR's laryngealized vowels in \textit{za'a} `corncob' and \textit{xa'ag} `topil'}
	\label{fig:FSRLaryngeal}
\end{figure}

The other consultant only ever produces creaky voice for these vowels regardless of the tone of the word. During one of the elicitation sessions, we conducted a perceptual check that these were, in fact, the same vowels, and both consultants reliably identified the words and produced laryngealized vowels according to their own idiosyncrasies. However, a more detailed perception study is beyond the scope of this paper. 
\begin{figure}[!h]
	\centering
	\begin{subfigure}{.5\textwidth}
		\centering
		\includegraphics[width=\linewidth]{Images/RD_za'a.png}
		\caption{\textit{za'a} `corncob'}
		\label{fig:za'a}
	\end{subfigure}%
	\begin{subfigure}{.5\textwidth}
		\centering
		\includegraphics[width=\linewidth]{Images/RD_xa'ag.png}
		\caption{\textit{xa'ag} `topil'}
		\label{fig:xa'ag}
	\end{subfigure}
	\caption{Comparison of RD's laryngealized vowels in \textit{za'a} `corncob' and \textit{xa'ag} `topil'}
	\label{fig:RDLaryngeal}
\end{figure}

%------------------------------------
\subsection{Interaction of Tone and Phonation} \label{sec:Interaction}
%------------------------------------

Most previous work on the interaction of tone has been focused on the languages of East and Southeast Asia \citep[e.g.,][]{masicaDefiningLinguisticArea1976,thurgoodVietnameseTonogenesisRevising2002,yipTone2002,enfieldArealLinguisticsMainland2005,michaudComplexTonesEast2012,brunelleTonePhonationSoutheast2016}. What has been found in these descriptions is that certain tones and phonations are codependent. For example, \citet{smalleyProblemsConsonantsTone1976} and \citet{ratliffMeaningfulToneStudy1992} both describe White Hmong's \textit{-g} tone as being a mid-low tone with breathy phonation and Mandarin's tone 3 is often associated with creaky phonation \citep{hockettPeipingPhonology1947}. \citet{brunelleTonePerceptionNorthern2009} found that creaky phonation plays an important role in producing certain tones. Additionally, work on S'gaw Karen has found that two tones are only differentiated by the presence of some form of non-modal phonation (Boehm p.c.). 

However, there have been some observations–especially in Mesoamerica–that tone and phonation are independent of each other \citep[e.g.,][]{silvermanLaryngealComplexityOtomanguean1997,garellekAcousticConsequencesPhonation2011}. This means that tone can independently occur with any phonation type. This has also been extensively described in multiple Zapotecan languages \citep[e.g.,][]{,avelinobecerraTopicsYalalagZapotec2004,avelinoAcousticElectroglottographicAnalyses2010, chavez-peonInteractionMetricalStructure2010, campbellZenzontepecChatinoAspect2011,villardPhonologyMorphologyZacatepec2015, lopeznicolasEstudiosFonologiaGramatica2016}

\citet{chavez-peonInteractionMetricalStructure2010} describes the tone and phonation interactions in San Lucas Quiaviní Zapotec (SLQZ), a central valley variety of Zapotec. The distribution of tone and phonation is found in Table~\ref{tab:SLQZ}. We see that in SLQZ, both low- and falling-tones have the full range of possible combinations. However, we see gaps in the high-tone for breathy and rising tones that can only occur with modal phonation. 

\begin{table}[!ht]
	\centering
	\caption{SLQZ tone and phonation interactions \citep{chavez-peonInteractionMetricalStructure2010}.}
	\label{tab:SLQZ}
	 \begin{tabular}{lcccc}
	  \lsptoprule
					  &	 \textbf{Modal}  & \textbf{Breathy} & \textbf{Creaky} & \textbf{Interrupted} \\
		  High	& ✔︎ & -- & ✔︎ & ✔︎ \\
		  Low & ✔︎ & ✔︎ & ✔︎ & ✔︎ \\
		  Falling & ✔︎ & ✔︎ & ✔︎ & ✔︎ \\
		  Rising & ✔︎ & -- & -- & -- \\
	  \lspbottomrule
	 \end{tabular}
\end{table}

Based on elicitation data collected from 2020-2022, SLZ has a more expansive distribution of tone and phonation when compared to SLQZ but seems to be very similar to other Northern Zapotec varieties \citep[e.g.,][]{avelinobecerraTopicsYalalagZapotec2004}. The distribution of SLZ tonal and phonation interactions are given in Table~\ref{tab:ToneVoiceQuality} for the number of analyzed tokens. 
\begin{table}[!h]
	\caption{SLZ tone and phonation interactions.}
	\label{tab:ToneVoiceQuality}
	\centering

	\begin{tabular}{lcccc}
	\lsptoprule
		& \textbf{Modal} & \textbf{Breathy} & \textbf{Checked} & \textbf{Laryngealized} \\
	\hline
	High		& ✔︎ & -- & ✔︎ & ✔︎ \\
	Mid			& ✔︎ & ✔︎ & ✔︎ & ✔︎ \\
	Low			& ✔︎ & ✔︎ & ✔︎ & ✔︎ \\
	High-Low	& ✔︎ & ✔︎ & ✔︎ & ✔︎ \\
	Mid-High	& ✔︎	& ✔︎ & -- & ✔︎ \\
	\lspbottomrule
	\end{tabular}
\end{table}

One of the striking things in this is the lack of high tone with breathy phonation. This gap is interesting because of the long-time association of high pitch with breathiness \citep[a good overview–of this association and other phonation types–is found in][]{eslingVoiceQualityLaryngeal2019}. This breathy phonation and high tone gap are common across the Zapotecan languages (Campbell p.c.). Regarding breathy phonation in SLQZ, \citet{uchiharaToneRegistrogenesisQuiavini2016} offers some convincing evidence that the phonation originated in syllables with low tone and then spread to other tones via analogy. If this is also true for SLZ, then we should be able to find similar distributions that Uchihara observes when we compare SLZ to its closest relative, San Bartolomé Zoogocho Zapotec, which lacks a breathy phonation. Such an investigation would be important in understanding how breathy vowels originated in the Zapotecan family, but is beyond the scope of this paper.  

%------------------------------------
\section{Methodology} \label{sec:Methods}
%------------------------------------

18 native language speakers of SLZ participate in this study (seven male). However, due to issues with annotation, I present data from 6 speakers (three male). All speakers live in Santiago Laxopa Cruz, Ixtlán, Oaxaca, Mexico. The data was collected using a Zoom H4n handheld recorder (44.1kHz, 16-bit). Participants were recorded saying approximately 76 words in isolation, and the carrier sentence \textit{shnia' X chone las} `I say X three times'. This phrase was repeated three times. 

%------------------------------------
\subsection{Acoustic measurements} \label{sec:Acoustics}
%------------------------------------

The vowels from the target words in carrier sentences were annotated in Praat \citep{boersmaPraatDoingPhonetics2021}. Following \citet{garellekAcousticDiscriminabilityComplex2020}, the vowel onset was set to the second glottal pulse following the onset (to avoid high-frequency noise due to noisy click releases), and the vowel offset was set to the last glottal pulse before the drop in amplitude of the following consonant. 

VoiceSauce \citep{shueVOICESAUCEProgramVoice2009} was used to compute the different acoustic measurements across each vowel. Each vowel was resampled at 16kHZ. The fundamental frequency (\textit{f0}) was calculated using STRAIGHT \citep{kawaharaInstantaneousfrequencybasedPitchExtraction1998}. Formants and bandwidths where calculated using SNACK \citep{sjolanderSnackSoundToolkit2004}.

A total of 1785 vowels were initially considered for analysis. Outliers were removed based on \textit{f0} and vowel formants distance. A total of 28 vowels whose \textit{f0} had a mahalanobis distance greater than 6 were removed as outliers. An additional 85 vowels were removed as outliers because their F1 and F2 had a standard deviation greater than 3. This resulted in a total of 1672 vowels being used for this analaysis.

Because phonation in SLZ involves dynamic changes (i.e., glottalization occurring in the middle or end of the vowel for laryngealized and checked vowels) each acoustic parameter had three different measurements associated with it. These three measures are: the average over the entire vowel, the change in the parameter from the beginning to the middle of the vowel, and the change in the parameter from the middle of the vowel to the vowel’s end. These last two measurements will be called ``delta" measurements. This method is not that unusual and was used by \citet{garellekAcousticDiscriminabilityComplex2020} for the phonation contrasts in !Xóõ.  

%------------------------------------
\section{Results} \label{sec:Results}
%------------------------------------

%------------------------------------
\subsection{Linear Discriminant Analysis} \label{sec:LDAResults}
%------------------------------------

A Linear Discriminant Analysis (LDA: \cite{fisherUseMultipleMeasurements1936}) is a type of dimension reduction technique in statistics. What this means in practice is that an LDA takes multiple variables as the input to the model and reduces those multiple variables into a few new variables called linear discriminant functions (LD). These LDs are then used to identify which of the original variables have the greatest effect on the data. For this study, an LDA was conducted in R \citep{rcoreteamLanguageEnvironmentStatistical2021} using the MASS package from \citet{venablesModernAppliedStatistics2002}.

In the case of SLZ phonation, each of the different acoustic measurements from \posscitet{kreimanUnifiedTheoryVoice2014} were treated as the variables that the LDA used to determine the new LDs. For the statistical model to not be unduly swayed by large numerical differences in the raw numbers it is standard that all variables are normalized. In this instance, all acoustic measurements proposed by the PMoP were normalized by speaker using a z-score transformation except for the measurements for formants and bandwidths which were normalized by speaker and vowel quality. Strength-of-Excitation (SoE) was normalized by speaker following \citet{garellekVoicingGlottalConsonants2021}. 

The number of LDs produced by the model is always the number of categories you are trying to predict minus one. Because SLZ has four phonation types, the LDA produced three LDs which accounted for 100\% of the variation in the data. The first linear discriminant (LD1) accounted for 53.24\% of the variation while the second (LD2) and third (LD3) linear discriminant functions accounted for 30.73\% and 16.03\% respectively. The confusion matrix from the LDA is shown in Table~\ref*{tab:Confusion}; overall, the accuracy of the model's classification was high, at 70.45\%. 

\begin{table}[!h]
    \centering
    \caption{Confusion matrix from the linear discriminate analysis.}
    \label{tab:Confusion}
    \begin{tabular}{lllll}
	\lsptoprule
	\begin{tabular}[c]{@{}l@{}}Actual → \\ Predicted ↓\end{tabular} & modal & breathy & checked & laryngealized \\
	\hline
	modal & 294 & 28 & 25 & 24 \\
	breathy & 16 & 32 & 9 & 15 \\
	checked & 13 & 5 & 22 & 3 \\
	laryngealized & 4 & 6 & 3 & 12\\
    \lspbottomrule
    \end{tabular}
\end{table}

Modal vowels represent 64\% of the training data while breathy vowels represent 15.2\%, checked vowels represent 10.8\%, and laryngealized vowels represent 10\%.  Modal vowels were the most accurately discriminated (89.9\%), whereas laryngealized vowels were the least accurately discriminated (22.2\%). All phonation types are most confusable with modal ones. Additionally, laryngealized vowels were frequently confused with breathy vowels.

Within each LD, the different acoustic measures are assigned a coefficient which indicates the strength that measure has on the LD. When evaluating which measures had the greatest strength you take the absolute value of the coefficient and look for the largest coefficients. The results of this evaluation produced the results reported in Table~\ref*{tab:LDA} (only the three strongest measures are show for each LD).
\begin{table}[!h]
    \centering
    \caption{Acoustic measures best captured by the three discriminant functions.}
    \label{tab:LDA}
    \begin{tabular}{lll}
	\lsptoprule
	LD1  & LD2 & LD3  \\
	\hline
    \begin{tabular}[c]{@{}l@{}}HNR \textless 2500 Hz\\ HNR \textless 3500 Hz\\ SoE\end{tabular} & \begin{tabular}[c]{@{}l@{}}HNR \textless 2500 Hz\\ HNR \textless 500 Hz\\ SoE\end{tabular} & \begin{tabular}[c]{@{}l@{}}HNR \textless 2500 Hz\\ HNR \textless 3500 Hz\\ $\Delta$HNR \textless 2500 Hz (end)\end{tabular} \\
    \lspbottomrule
    \end{tabular}
\end{table}

LD1 is used to discriminant modal vowels from the other phonation types and is most strongly associated with HNR \textless 2500 Hz, HNR \textless 3500 Hz, and SoE; see Figure~\ref*{fig:LD1}. 
\begin{figure}[!h]
	\centering
	\includegraphics[width=\linewidth]{Images/ld1_density.png}
	\caption{Density plot showing LD1 grouped by SLZ's phonation types.}
	\label{fig:LD1}
\end{figure}

LD2 is used to discriminate breathy and checked vowels from the other phonation types and is also most strongly associated with HNR \textless 2500 Hz and SoE. Additionally, HNR \textless 500 Hz is also strongly associated with LD2; see Figure~\ref*{fig:LD2}. 
\begin{figure}[!h]
	\centering
	\includegraphics[width=\linewidth]{Images/ld2_density.png}
	\caption{Density plot for LD2 grouped by SLZ's phonation types.}
	\label{fig:LD2}
\end{figure}

LD3 is used to discriminate laryngealized vowels from the other phonation types and is also highly associated with the same measures as LD1 except for SoE which is instead replaced by $\Delta$HNR \textless 2500 Hz (end) which represents the difference in HNR \textless 2500 Hz from the middle to the end of the vowel; see Figure~\ref*{fig:LD3}. 
\begin{figure}[!h]
		\centering
		\includegraphics[width=\linewidth]{Images/ld3_density.png}
		\caption{Density plot for LD3 grouped by SLZ's phonation types.}
		\label{fig:LD3}
\end{figure}


%------------------------------------
\subsection{Statistical results on the output of the LDA} \label{sec:Stats}
%------------------------------------

The five resulting acoustic measures from the LDA were then analyzed for statistical significance through a mixed-effects linear regression using R's lme4 package \citep{batesFittingLinearMixedEffects2015}. In each linear regression model, the acoustic measures were treated as the dependent variable with phonation as the predictor the interaction between speaker and phonation was treated as a random variable as was each of the token words.\footnotetext{DV \thicksim Phonation + (Phonation|Speaker) + (1|Word)} When evaluating the mixed-effects linear regression models the modal vowel was also treated as the intercept of the model. This means that the results of each non-modal phonation type are made in reference to the modal vowel.  

The first linear discriminant function was most strongly affected by HNR\textless 2500 Hz, HNR\textless 3500 Hz, and Strength-of-Excitation. The two HNR measures are associated with measures of periodicity but in different bandwidths. HNR\textless 2500 Hz is concerned with measuring periodicity below 2500 Hz. Because non-modal phonation is associated with aperiodicity we expect the values for breathy, checked, and laryngealized vowels to be lower than the values for modal vowels. As we see in Figure~\ref*{fig:HNR25}, the values for breathy, checked, and laryngealized vowels are all lower than the modal vowels. 
\begin{figure}[!h]
	\centering
	\includegraphics[width=\linewidth]{Images/HNR25.png}
	\caption{Line graph showing the change of HNR\textless 2500 Hz across the vowels grouped by SLZ's phonation types.}
	\label{fig:HNR25}
\end{figure}

When we turn to the mixed-effects linear regression model for HNR<2500 Hz, we observe that for every step of change in modal vowels all three values will change negatively with respect to the modal's values. However, only the changes in breathy and checked vowels are significant; see Table~\ref*{tab:HNR25}.
\begin{table}[!h]
    \centering
    \caption{Results of the mixed-effects linear regression analysis for HNR < 2500 Hz.}
    \label{tab:HNR25}
    \begin{tabular}{lrrrrrl}
	\lsptoprule
					&  Estimate  & Std. Error & df & t value & p-value & \\
    Modal       	&  0.1347	& 0.0558 & 151.5648 &  2.414 & 0.01696 & * \\  
  	Breathy   		&  -0.4914 	& 0.1530 &  9.9072  & -3.213 & 0.00939 & ** \\
	Checked    		&  -0.4669  & 0.1769 &  6.6601  & -2.640 & 0.03499 & * \\
	Laryngealized	&  -0.1654  & 0.1810 & 17.8292  & -0.914 & 0.37293 & \\
    \lspbottomrule
    \end{tabular}
\end{table}

When we turn to the acoustic measure HNR<3500 Hz, which measures periodicity below 3500 Hz, we see that the three non-modal phonation types all show lower values than modal vowels; see Figure~\ref*{fig:HNR35}. Similar to HNR<2500 Hz, these lower HNR values indicate that these phonation types show aperiodicity in their production. 
\begin{figure}[!h]
	\centering
	\includegraphics[width=\linewidth]{Images/HNR35.png}
	\caption{Line graph showing the change of HNR\textless 3500 Hz across the vowels grouped by SLZ's phonation types.}
	\label{fig:HNR35}
\end{figure}

In contrast to the mixed-effect linear regression model for HNR<2500 Hz, the mixed-effects linear regression for HNR\textless 3500 Hz shows that the differences we observe between the non-modal phonation types and the modal vowels are not statistically significant; see Table~\ref*{tab:HNR35}. 
\begin{table}[!h]
    \centering
    \caption{Results of the mixed-effects linear regression analysis for HNR < 3500 Hz.}
    \label{tab:HNR35}
    \begin{tabular}{lrrrrrl}
	\lsptoprule
					&  Estimate  & Std. Error & df & t value & p-value & \\
    Modal       	&   0.10018  &  0.05403 & 150.80549 &  1.854 &  0.0657 & . \\  
  	Breathy   		&  -0.33967  &  0.15350 &  9.77806  &-2.213 &  0.0519 & . \\
	Checked    		&  -0.37542  &  0.17313 &  6.62514  &-2.168 &  0.0690 & . \\
	Laryngealized	&  -0.09672  &  0.16999 & 20.16718  &-0.569 &  0.5757 & \\
    \lspbottomrule
    \end{tabular}
\end{table}

The last acoustic measure that was strongly associated with the first linear discriminant function was Strength-of-Excitation which is a measure of the intensity of voicing regardless of whatever noise might be in the signal. This SoE measure means that the closer the value is to zero the less intense the voicing is and indicates a constriction closure to the glottis \citep{mittalStudyEffectsVocal2014,garellekVoicingGlottalConsonants2021}. 

\begin{figure}[!h]
	\centering
	\includegraphics[width=\linewidth]{Images/SoE.png}
	\caption{Line graph showing the change of Strength-of-Excitation across the vowels grouped by SLZ's phonation types.}
	\label{fig:SOE}
\end{figure}

\begin{table}[!h]
    \centering
    \caption{Results of the mixed-effects linear regression analysis for SoE.}
    \label{tab:SOE}
    \begin{tabular}{lrrrrrl}
	\lsptoprule
					&  Estimate  & Std. Error & df & t value & p-value & \\
    Modal       	&   0.65006  &  0.04328 &  5.33922 & 15.018 & 1.43e-05 & *** \\  
  	Breathy   		&  -0.11532  &  0.03057 &  7.41179 & -3.772 & 0.006265 & ** \\
	Checked    		&  -0.11516  &  0.02230 &  9.38943 & -5.165 & 0.000517 & *** \\
	Laryngealized	&  -0.08573  &  0.03299 & 11.87923 & -2.599 & 0.023424 & * \\
    \lspbottomrule
    \end{tabular}
\end{table}

\begin{figure}[!h]
	\centering
	\includegraphics[width=\linewidth]{Images/HNR05.png}
	\caption{Line graph showing the change of HNR < 500 Hz across the vowels grouped by SLZ's phonation types.}
	\label{fig:HNR05}
\end{figure}

\begin{table}[!h]
    \centering
    \caption{Results of the mixed-effects linear regression analysis for HNR < 500 Hz.}
    \label{tab:HNR05}
    \begin{tabular}{lrrrrrl}
	\lsptoprule
					&  Estimate  & Std. Error & df & t value & p-value & \\
    Modal       	&  0.14980  & 0.07553 & 9.64369 &  1.983 & 0.07652 & . \\  
  	Breathy   		&  -0.05906 & 0.20426 & 6.38671 & -0.289 & 0.78162 &  \\
	Checked    		&  -0.70904 & 0.16811 & 6.77873 & -4.218 & 0.00424 & ** \\
	Laryngealized	&  -0.29020 & 0.23947 & 7.58130 & -1.212 & 0.26197 & \\
    \lspbottomrule
    \end{tabular}
\end{table}


\begin{table}[!h]
    \centering
    \caption{Results of the mixed-effects linear regression analysis for $ \Delta $ HNR<2500 Hz (end).}
    \label{tab:delta}
    \begin{tabular}{lrrrrrl}
	\lsptoprule
					&  Estimate  & Std. Error & df & t value & p-value & \\
    Modal       	&   -0.02397 &   0.06574  & 32.69794 & -0.365 & 0.71774 & \\  
  	Breathy   		&    0.12279 &   0.14446  & 11.41768 &  0.850 & 0.41280 & \\
	Checked    		&    0.32531 &   0.10589  & 41.57481 &  3.072 & 0.00374 & ** \\
	Laryngealized	&   -0.12616 &   0.27805  &  8.21579 & -0.454 & 0.66176 & \\
    \lspbottomrule
    \end{tabular}
\end{table}

%------------------------------------
\section{Discussion} \label{sec:Discussion}
%------------------------------------

Based on the results of the spectral-tilt measurements, there is some agreement and disagreement with established patterns in what measures capture the different phonations best. 

Contrary to \citet{espositoVariationContrastivePhonation2010}, H1-H2 was not the best measurement for all phonation contrasts for the female SLZ speaker. FSR's breathy vowels were best measured using H1-A3. This fact suggests several things. We can first surmise that H1-H2 is not always the best measurement for phonation. This aligns with more recent findings that H1-H2 is a poor measure of phonation in general \citep{chaiH1H2Acoustic2022}. H2 is sensitive to differences in subglottal pressure \citep{sundbergObjectiveCharacterizationPhonation2022}. We can surmise this because breathy phonation had a very low value, which is opposite from what is commonly expected, a higher value than the modal's value. Additionally, the only time this measure behaved in a way that we expected was for FSR's checked vowels. In the final third of the vowel, the lower H1-H2 measure indicates a period of creakiness that corresponds with the glottal closure accompanying these vowels. 

This behavior of H1-H2 also suggests that \posscitet{espositoVariationContrastivePhonation2010} observation that H1-H2 is best for female speakers of Santa Ana del Valle Zapotec and H1-A3 for males might not apply to all varieties of Zapotec. However, there remains a large amount of work to confirm the results of this study. Primarily, this work was based on the results of only two speakers. More speakers can be consulted now that the COVID-19 pandemic has decreased in severity, and most of Santiago Laxopa has been vaccinated. This is especially important because it will allow researchers to understand better how laryngealized vowels are produced and what cues help differentiate them. 

This fact concerning laryngealized vowels is especially true given that FSR's spectral-tilt measurements for laryngealized vowels were nearly identical to those for modal vowels. However, one important thing is that we still got a lower CPP vowel in the second-third of the vowel in these laryngealized vowels. This position is precisely where one would expect acoustic cues for aperiodic noise. Additionally, we see in the spectrograms for these laryngealized vowels a decrease in amplitude for some time. This decrease in amplitude is significant because \citet{gerfenProductionPerceptionLaryngealized2005} found that even a slight reduction in amplitude was enough for people to identify a glottal stop. When we consider RD's measurements for laryngealized vowels, which RD consistently produces with creaky voice, we find that this vowel shows a consistently lower H1-A3 beginning in the middle of the vowel and continuing throughout the rest of the vowel. This pattern matches what one would expect to see.

These findings answer some questions but raise several others. The questions it answers concern the most reliable cues to phonation. It shows that languages are not uniform in relying on the same spectral-tilt measurements and types of harmonics-to-noise but that each language may make use of different types and combinations of these measurements to cue phonation, which seems to be somewhat more analogous to the varying ways in which languages realize voicing contrasts \citep{liskerCrossLanguageStudyVoicing1964}. It is well known that English's voicing contrast is a difference in voiceless plain stops and voiceless aspirated stops in word-initial position. In contrast, languages like Dutch and Spanish make an actual voicing distinction. 

Another question has to do with the predictions from the Laryngeal Complexity Hypothesis. The \textsc{Laryngeal Complexity Hypothesis} (LCH) has its origins in work by \citet{silvermanLaryngealComplexityOtomanguean1997,blankenshipTimeCourseBreathiness1997,blankenshipTimingNonmodalPhonation2002}. The basic premise of the LCH is that in languages that have both tone and phonation, there needs to be an ordering between the two laryngeal gestures for tone and phonation for optimal perceptibility. This premise comes from the understanding that the same mechanism responsible for tone is also responsible for the production of phonation: the vocal folds and glottis. The rate at which vocal folds vibrate is responsible for changing the fundamental frequency, perceived as pitch, which is grammaticalized as tone in lexical tone languages. The faster the vocal folds vibrate, the higher the pitch. And the slower the vocal folds vibrate, the lower the pitch. 

The vocal folds are also the primary articulator for phonation. \citet{ladefogedPreliminariesLinguisticPhonetics1971,gordonPhonationTypesCrosslinguistic2001} treated phonation as a by-product of how open or closed the glottis is during vocal fold vibration. This is schematized by Figure~\ref{fig:Phonation}. The more open the glottis is, the breathier the sound to the point where the sound becomes the sound [h]. The more closed the glottis is, the creakier the sound is to the point where the sound becomes [ʔ]. 

\begin{figure}[!ht]
	\centering
	\includegraphics[width=.6\textwidth]{Phonation.png}
	\caption{Simplified one-dimensional model for phonation. Based on \citet{ladefogedPreliminariesLinguisticPhonetics1971,gordonPhonationTypesCrosslinguistic2001}}.
	\label{fig:Phonation2}
\end{figure}

The LCH assumes phonation and tone are produced at the vocal folds and glottis. Because these same organs are responsible for these two phenomena, a mismatch exists in trying to have both simultaneously. It is assumed by the LCH that because of this issue, there needs to be strict ordering in the glottal gestures. This means the tonal gesture must be produced before or after the phonation gesture. The reason for this is that if the gestures overlap, there will be a perturbation of the tone, and the listeners will not be able to differentiate the tone reliably. The LCH assumes that there is a close link between production and perception. This assumption places the responsibility on making sure the acoustic cues are the most perceptually salient on the speaker. The speaker is responsible for producing both tone and phonation so that the listener can differentiate the different cues for both tone and phonation. These assumptions can best be represented by Figure~\ref{fig:GlottalGestures}. 

In Figure~\ref{fig:GlottalGestures}, which is taken from \citet{dicanioCoarticulationToneGlottal2012}, the cue for tone is represented by the Pitch Target and the Glottal Gesture represents the gestures needed to produce phonation. When the Pitch Target is not co-articulated with the Glottal Gestures, there is the greatest perceptual recoverability for the listener. The LCH argues for this as the tones are the most recoverable on modal vowels. This modal portion is then ordered or phased relative to the non-modal portion. This ordering is the speaker's responsibility to accommodate the listeners' perceptibility. If, however, the Pitch Target and the Glottal Gesture were overlapping, as represented by the lower half in Figure~\ref{fig:GlottalGestures}, then the cues for pitch and phonation would be overlapping and making it perceptually more difficult for 
\begin{figure}[!ht]
	\centering
	\includegraphics[width=.5\textwidth]{Gestures.png}
	\caption{Representation taken from \citet{dicanioCoarticulationToneGlottal2012}.}
	\label{fig:GlottalGestures}
\end{figure}

Some work has been done investigating this in other languages, most notably \posscitet{dicanioCoarticulationToneGlottal2012} investigation into glottals in Itunyoso Trique, which is also an Oto-Manguean language. \citeauthor{dicanioCoarticulationToneGlottal2012} in his study found that when the magnitude of coarticulation for glottal consonants occurs on the vowels, there is a strong correlation between the magnitude of overlap and the amount of perturbation in the f0 signal. If the degree of overlap was minor, then the acoustic signal had little to no perturbation. These results were found by consulting the spectral tilt of the vowels with the f0 measures and performing a generalized linear mixed effects model with the speaker treated as a random effect. 

Another study on Jalapa Mazatec \citep{garellekAcousticConsequencesPhonation2011} also investigated the interaction of tone and phonation. Jalapa Mazatec is a language with both contrastive tone and phonation, and \citet{garellekAcousticConsequencesPhonation2011} validated the claims made by the LCH, in that tone and phonation seemed to be ordered with each other when it comes to at least one of the phonation types. 

This paper describes the tonal and phonation inventories in Santiago Laxopa Zapotec, an ideal language for testing the viability of the LCH because of its use of both contrastive tone and contrastive phonation.   

Another question this study raises is the status of laryngealized vowels and their realizations. We observed from just two speakers that these vowels are highly variable. Assuming that \posscitet{avelinoAcousticElectroglottographicAnalyses2010} observations on this vowel in the closely related Yalálag Zapotec hold in SLZ, there is considerably more variation that exists within and between speakers. Suppose all of these varying realizations of a phonological category are evident. What does this mean for this category's status and the cues necessary for speakers to produce and realize this category? 

%------------------------------------
\section{Further directions and Conclusions} \label{sec:Conclusion}
%------------------------------------

In conclusion, this paper has briefly introduced the tonal and phonation systems of Santiago Laxopa Zapotec, an understudied variety of Sierra Norte Zapotec. This system is essential for exploring the acoustic cues used in different languages to differentiate each phonation type.

Contrary to acoustic work on other Zapotec varieties by \citet{espositoVariationContrastivePhonation2010}, which showed that female speakers' phonation contrasts were best characterized by H1-H2 and male speakers' contrasts by H1-A3, SLZ suggests that both male and female speakers' breathy voice is best described by H1-A3 and checked and laryngealized voice are characterized by a difference in the timing of spectral-tilt measurements and CPP values depending on the speaker. 

Because of these phonation type differences in behavior, one could, in theory, speak to how SLZ compensates for using the larynx for both tone and phonation. This paper has not explicitly spoken on this issue, and the Laryngeal Complexity Hypothesis presented by \citet{silvermanLaryngealComplexityOtomanguean1997} and \citet{blankenshipTimeCourseBreathiness1997, blankenshipTimingNonmodalPhonation2002}. The LCH states that when a language has tone and phonation, these different aspects are ordered or phased concerning one another. This phasing allows for the most significant perceptibility in the acoustic signal. This allows the listener to interpret the acoustic signals for tone and phonation adequately. 

Overall, there seems to be some information from the spectral-tilt and CPP analysis that speaks to the question of the LCH. If the reader recalls, CPP was the most important cue for differentiating checked and laryngealized vowels from modal and breathy vowels for FSR. The key is the lower CPP value, which corresponds to greater aperiodicity. It is worth repeating that we did see a difference in the timing of the aperiodicity between checked and laryngealized vowels. In laryngealized vowels, this dip in CPP occurs in the middle third of the vowel, and for checked vowels, this dip occurs in the final third. However, the fact that for breathy vowels, the most noticeable cue is the higher H1-A3 which happens throughout the entire vowel, might suggest that breathiness does not behave in the same way as the other phonation types. The behavior from checked and laryngealized vowels seems to support the LCH. Unfortunately, no firm conclusions can be made about the LCH, and further investigation is required to determine the validity of the LCH in SLZ. 

This ability of speakers to interpret the acoustic signals in conjunction with the acoustic signals for tone is of great interest. It would benefit from a perception experiment to determine what the speakers are using to differentiate the different phonation types. This is especially true for laryngealized vowels, which have such varied pronunciations. 

This study will benefit from further analysis and data collection. Now that the world is safer regarding COVID-19, collecting data from more speakers is essential to corroborate the data and analysis from FSR and RD. 

%------------------------------------
%BIBLIOGRAPHY
%------------------------------------

%\singlespacing
% \nocite{*}
\printbibliography[heading=bibintoc]

\end{document}